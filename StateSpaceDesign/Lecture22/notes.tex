%% State Space Modelling of Dynamic Systems
%% Lecture 22: Combined Observer and Control
\def\FileDate{10/04/02}
\def\FileVersion{1.0}
% ----------------------------------------------------------------
% Notes pages *********************************************************
% ----------------------------------------------------------------

Therefore the dynamics of the combined system are:
\[
\left[ {\begin{array}{*{20}c}
   {{\bf{\dot x}}}  \\
   {{\bf{\dot e}}}  \\
\end{array}} \right] = \left[ {\begin{array}{*{20}c}
   {\left( {{\bf{A}} - {\bf{BK}}} \right)} & {{\bf{BK}}}  \\
   {\bf{0}} & {\left( {{\bf{A}} - {\bf{LC}}} \right)}  \\
\end{array}} \right]\left[ {\begin{array}{*{20}c}
   {\bf{x}}  \\
   {\bf{e}}  \\
\end{array}} \right] + \left[ {\begin{array}{*{20}c}
   {\bf{B}}  \\
   {\bf{0}}  \\
	\end{array}} \right]r
\]

\endinput

%%% Local Variables: 
%%% mode: latex
%%% TeX-master: "notes"
%%% End:
\ifslidesonly
\begin{slide}
	\heading{Combined Observer and Control}
   Therefore the dynamics of the combined system are:
\[
\left[ {\begin{array}{*{20}c}
   {{\bf{\dot x}}}  \\
   {{\bf{\dot e}}}  \\
\end{array}} \right] = \left[ {\begin{array}{*{20}c}
   {\left( {{\bf{A}} - {\bf{BK}}} \right)} & {{\bf{BK}}}  \\
   {\bf{0}} & {\left( {{\bf{A}} - {\bf{LC}}} \right)}  \\
\end{array}} \right]\left[ {\begin{array}{*{20}c}
   {\bf{x}}  \\
   {\bf{e}}  \\
\end{array}} \right] + \left[ {\begin{array}{*{20}c}
   {\bf{B}}  \\
   {\bf{0}}  \\
	\end{array}} \right]r
\]

\endinput

%%% Local Variables: 
%%% mode: latex
%%% TeX-master: "notes"
%%% End:
\end{slide}
\fi

\ifslidesonly
\begin{slide}
	\heading{Contents of this Lecture}
   \begin{itemize}
   	\item Dynamics of the combined controller-observer system
\begin{itemize}
	\item Separation principle
	\item Equivalent classical controller
\end{itemize}
\item Introducing the reference input
\begin{enumerate}
	\item The normal position
	\item Observer driven by the tracking error
\end{enumerate}
   \end{itemize}
\end{slide}
\fi

\section*{Dynamics of the combined system} % (fold)
\label{sub:dynamics_of_the_combined_system}

In particular, the solution of:
\[
\frac{d\mathbf{x}}{dt}=\mathbf{Ax}
\]
given $\mathbf{x}=\mathbf{x}_0$ at $t=0$ is:
\[
\mathbf{x}=e^{\mathbf{A}t}\mathbf{x}_0
\]
               
The term  $\mathbf{\phi}(t) = e^{\mathbf{A}t}$  is known as the state transition matrix because it shows how time, $t$, transforms the initial state vector into the present one.

\endinput

%%% Local Variables: 
%%% mode: latex
%%% TeX-master: "notes"
%%% End:
\ifslidesonly
\begin{slide}
	\heading{Dynamics of the combined system (1)}
   In particular, the solution of:
\[
\frac{d\mathbf{x}}{dt}=\mathbf{Ax}
\]
given $\mathbf{x}=\mathbf{x}_0$ at $t=0$ is:
\[
\mathbf{x}=e^{\mathbf{A}t}\mathbf{x}_0
\]
               
The term  $\mathbf{\phi}(t) = e^{\mathbf{A}t}$  is known as the state transition matrix because it shows how time, $t$, transforms the initial state vector into the present one.

\endinput

%%% Local Variables: 
%%% mode: latex
%%% TeX-master: "notes"
%%% End:
\end{slide}
\fi

In the control canonical form we have matrices:
% MathType!MTEF!2!1!+-
% faaagaart1ev2aaaKnaaaaWenf2ys9wBH5garuavP1wzZbqedmvETj
% 2BSbqefm0B1jxALjharqqtubsr4rNCHbGeaGqiVu0Je9sqqrpepC0x
% bbL8FesqqrFfpeea0xe9Lq-Jc9vqaqpepm0xbba9pwe9Q8fs0-yqaq
% pepae9pg0FirpepeKkFr0xfr-xfr-xb9Gqpi0dc9adbaqaaeGaciGa
% aiaabeqaamaabaabaaGcbaGaaCyqaiabg2da9maadmaabaqbamqabq
% abaaaaaeaacqGHsislcaWGHbWaaSbaaSqaaiaaigdaaeqaaaGcbaGa
% eyOeI0IaamyyamaaBaaaleaacaaIYaaabeaaaOqaaiabl+Uimbqaai
% abgkHiTiaadggadaWgaaWcbaGaamOBaaqabaaakeaacaaIXaaabaGa
% aGimaaqaaiabl+UimbqaaiaaicdaaeaacaaIWaaabaGaeSy8I8eaba
% GaeS47IWeabaGaeSO7I0eabaGaaGimaaqaaiaaicdaaeaacqWIVlct
% aeaacaaIWaaaaaGaay5waiaaw2faaiaacUdacaaMf8UaaCOqaiabg2
% da9maadmaabaqbamqabqqaaaaabaGaaGymaaqaaiaaicdaaeaacqWI
% UlstaeaacaaIWaaaaaGaay5waiaaw2faaaaa!556B!
\[
{\bf{A}} = \left[ {\begin{array}{*{20}c}
   { - a_1 } & { - a_2 } &  \cdots  & { - a_n }  \\
   1 & 0 &  \cdots  & 0  \\
   0 &  \ddots  &  \cdots  &  \vdots   \\
   0 & 0 &  \cdots  & 0  \\
\end{array}} \right];\quad {\bf{B}} = \left[ {\begin{array}{*{20}c}
   1  \\
   0  \\
    \vdots   \\
   0  \\
\end{array}} \right]
\]
with open loop CE:
\[
\det(s\mathbf{I}-\mathbf{A})=s^n+a_1s^{n-1}+\cdots+a_n=0.
\]

\endinput

%%% Local Variables: 
%%% mode: latex
%%% TeX-master: "notes"
%%% End:
\ifslidesonly
\begin{slide}
	\heading{Dynamics of the combined system (2)}
   In the control canonical form we have matrices:
% MathType!MTEF!2!1!+-
% faaagaart1ev2aaaKnaaaaWenf2ys9wBH5garuavP1wzZbqedmvETj
% 2BSbqefm0B1jxALjharqqtubsr4rNCHbGeaGqiVu0Je9sqqrpepC0x
% bbL8FesqqrFfpeea0xe9Lq-Jc9vqaqpepm0xbba9pwe9Q8fs0-yqaq
% pepae9pg0FirpepeKkFr0xfr-xfr-xb9Gqpi0dc9adbaqaaeGaciGa
% aiaabeqaamaabaabaaGcbaGaaCyqaiabg2da9maadmaabaqbamqabq
% abaaaaaeaacqGHsislcaWGHbWaaSbaaSqaaiaaigdaaeqaaaGcbaGa
% eyOeI0IaamyyamaaBaaaleaacaaIYaaabeaaaOqaaiabl+Uimbqaai
% abgkHiTiaadggadaWgaaWcbaGaamOBaaqabaaakeaacaaIXaaabaGa
% aGimaaqaaiabl+UimbqaaiaaicdaaeaacaaIWaaabaGaeSy8I8eaba
% GaeS47IWeabaGaeSO7I0eabaGaaGimaaqaaiaaicdaaeaacqWIVlct
% aeaacaaIWaaaaaGaay5waiaaw2faaiaacUdacaaMf8UaaCOqaiabg2
% da9maadmaabaqbamqabqqaaaaabaGaaGymaaqaaiaaicdaaeaacqWI
% UlstaeaacaaIWaaaaaGaay5waiaaw2faaaaa!556B!
\[
{\bf{A}} = \left[ {\begin{array}{*{20}c}
   { - a_1 } & { - a_2 } &  \cdots  & { - a_n }  \\
   1 & 0 &  \cdots  & 0  \\
   0 &  \ddots  &  \cdots  &  \vdots   \\
   0 & 0 &  \cdots  & 0  \\
\end{array}} \right];\quad {\bf{B}} = \left[ {\begin{array}{*{20}c}
   1  \\
   0  \\
    \vdots   \\
   0  \\
\end{array}} \right]
\]
with open loop CE:
\[
\det(s\mathbf{I}-\mathbf{A})=s^n+a_1s^{n-1}+\cdots+a_n=0.
\]

\endinput

%%% Local Variables: 
%%% mode: latex
%%% TeX-master: "notes"
%%% End:
\end{slide}
\fi 




The eigenvalues or poles of the combined system are the roots of the CE:
\begin{eqnarray*}
	\det \left[ {s{\bf{I}}_{2n}  - \left[ {\begin{array}{*{20}c}
	   {\left( {{\bf{A}} - {\bf{BK}}} \right)} & {{\bf{BK}}}  \\
	   {\bf{0}} & {\left( {{\bf{A}} - {\bf{LC}}} \right)}  \\
	\end{array}} \right]} \right] & = &  0 \\
	\det \left[ {\begin{array}{*{20}c}
	   {\left[ {s{\bf{I}} - \left( {{\bf{A}} - {\bf{BK}}} \right)} \right]} & { - {\bf{BK}}}  \\
	   {\bf{0}} & {\left[ {s{\bf{I}} - \left( {{\bf{A}} - {\bf{LC}}} \right)} \right]}  \\
	\end{array}} \right] & = & 0
\end{eqnarray*}

\endinput

%%% Local Variables: 
%%% mode: latex
%%% TeX-master: "notes"
%%% End:
\ifslidesonly
\begin{slide}
	\heading{Characteristic Equation (1)}
   The eigenvalues or poles of the combined system are the roots of the CE:
\begin{eqnarray*}
	\det \left[ {s{\bf{I}}_{2n}  - \left[ {\begin{array}{*{20}c}
	   {\left( {{\bf{A}} - {\bf{BK}}} \right)} & {{\bf{BK}}}  \\
	   {\bf{0}} & {\left( {{\bf{A}} - {\bf{LC}}} \right)}  \\
	\end{array}} \right]} \right] & = &  0 \\
	\det \left[ {\begin{array}{*{20}c}
	   {\left[ {s{\bf{I}} - \left( {{\bf{A}} - {\bf{BK}}} \right)} \right]} & { - {\bf{BK}}}  \\
	   {\bf{0}} & {\left[ {s{\bf{I}} - \left( {{\bf{A}} - {\bf{LC}}} \right)} \right]}  \\
	\end{array}} \right] & = & 0
\end{eqnarray*}

\endinput

%%% Local Variables: 
%%% mode: latex
%%% TeX-master: "notes"
%%% End:
\end{slide}
\fi

\[
{\bf{A}} - {\bf{BK}}  =  \left[ {\begin{array}{*{20}c}
	   {( - a_1  - k_1 )} & {( - a_2  - k_2 )} &  \cdots  & {( - a_n  - k_n )}  \\
	   1 & 0 &  \cdots  &  \vdots   \\
	    \vdots  &  \ddots  &  \cdots  &  \vdots   \\
	   0 & 0 &  \cdots  & 0  \\
	\end{array}} \right]
\]
therefore
\begin{equation}
	\label{eq:5}
	\det(s\mathbf{I}-\mathbf{A}+\mathbf{BK}) = s^n + ( - a_1  - k_1 )s^{n-1} + ( - a_2  - k_2 )s^{n-2} + \cdots +  ( - a_n  - k_n ) = 0.
\end{equation}

\endinput

%%% Local Variables: 
%%% mode: latex
%%% TeX-master: "notes"
%%% End:
\ifslidesonly
\begin{slide}
	\heading{Characteristic Equation (2)}
   \[
{\bf{A}} - {\bf{BK}}  =  \left[ {\begin{array}{*{20}c}
	   {( - a_1  - k_1 )} & {( - a_2  - k_2 )} &  \cdots  & {( - a_n  - k_n )}  \\
	   1 & 0 &  \cdots  &  \vdots   \\
	    \vdots  &  \ddots  &  \cdots  &  \vdots   \\
	   0 & 0 &  \cdots  & 0  \\
	\end{array}} \right]
\]
therefore
\begin{equation}
	\label{eq:5}
	\det(s\mathbf{I}-\mathbf{A}+\mathbf{BK}) = s^n + ( - a_1  - k_1 )s^{n-1} + ( - a_2  - k_2 )s^{n-2} + \cdots +  ( - a_n  - k_n ) = 0.
\end{equation}

\endinput

%%% Local Variables: 
%%% mode: latex
%%% TeX-master: "notes"
%%% End:
\end{slide}
\fi


% section dynamics_of_the_combined_system (end)




\subsection*{Separation Principle} % (fold)
\label{sub:separation_principle}

From the above work, we can conclude that the set of poles of the combined observer-controller system is the union of the set of closed loop controller poles and the set of observer poles.

The controller matrix  $\mathbf{K}$  is designed as before, as if the real plant states are going to be used for the feedback. This fixes the positions of the controller poles into the desired positions.

Then, quite independently, the matrix $\mathbf{L}$ is designed as before to fix the observer poles as required.

Using the observer states for the control feedback instead of the plant states does not affect the closed loop poles.

This is a fortunate situation and is known as the \textbf{separation principle}. The problems of controller design and observer design have been separated.

\ifslidesonly
\begin{slide}
	\heading{Separation Principle}
	\begin{itemize}
		\item Poles of the combined observer-controller system is the union of closed loop controller and observer poles.
		\item The controller matrix  $\mathbf{K}$  is designed as before.
		\item Quite independently, the matrix $\mathbf{L}$ is designed to fix the observer poles as required.
		\item Using the observer states for the control feedback instead of the plant states does not affect the closed loop poles.
	\end{itemize}
   This is a fortunate situation and is known as the \textbf{separation principle}. The problems of controller design and observer design have been separated.
\end{slide}
\fi
 
% section separation_principle (end)


\subsection*{The Equivalent Classical Compensator} % (fold)
\label{sec:the_equivalent_classical_compensator}


\begin{itemize}
	\item We shall determine the classical compensator TF which is equivalent to the combined observer-controller. 
	\item This is simpler to do if we remove the reference input, for the time being.
\end{itemize}
\begin{center}
	\resizebox{300pt}{!}{\includegraphics{pictures/fig1.pdf}}
\end{center}
\endinput

%%% Local Variables: 
%%% mode: latex
%%% TeX-master: "notes"
%%% End:
\ifslidesonly
\begin{slide}
	\heading{The Equivalent Classical Compensator}
	\begin{itemize}
	\item We shall determine the classical compensator TF which is equivalent to the combined observer-controller. 
	\item This is simpler to do if we remove the reference input, for the time being.
\end{itemize}
\begin{center}
	\resizebox{300pt}{!}{\includegraphics{pictures/fig1.pdf}}
\end{center}
\endinput

%%% Local Variables: 
%%% mode: latex
%%% TeX-master: "notes"
%%% End:
\end{slide}
\fi


Setting the reference input $r = 0$  gives:
\begin{itemize}
	\item Rule of thumb: observer poles can be faster than the controller poles (i.e. further from the origin) by a factor of 2 to 6. This makes the effect of the observer dynamics short-term and the overall response is dominated by the controller poles.
	\item If noise/disturbance is present this has an effect on the choice:
	\begin{description}
		\item[Process noise $w$:] $d\mathbf{x}/dt=\mathbf{Ax}+\mathbf{B}u+\mathbf{B}_1 w$
		\item[Sensor noise $v$:]  $y = \mathbf{C}x+v$
		\item[Observer:] $d\hat{\mathbf{x}}=\mathbf{A}\hat{\mathbf{x}}+\mathbf{B}u+\mathbf{L}(y-\mathbf{C}\hat{\mathbf{x}})$
		\item[Error $\mathbf{e}=\mathbf{x}-\hat{\mathbf{x}}$:] $d\mathbf{e}/dt=(\mathbf{A}-\mathbf{LC})\mathbf{e}+\mathbf{B}_1 w - \mathbf{L}v.$
	\end{description}
\end{itemize}

\endinput

%%% Local Variables: 
%%% mode: latex
%%% TeX-master: "notes"
%%% End:
\ifslidesonly
\begin{slide}
	\heading{When $r=0$}
   \begin{itemize}
	\item Rule of thumb: observer poles can be faster than the controller poles (i.e. further from the origin) by a factor of 2 to 6. This makes the effect of the observer dynamics short-term and the overall response is dominated by the controller poles.
	\item If noise/disturbance is present this has an effect on the choice:
	\begin{description}
		\item[Process noise $w$:] $d\mathbf{x}/dt=\mathbf{Ax}+\mathbf{B}u+\mathbf{B}_1 w$
		\item[Sensor noise $v$:]  $y = \mathbf{C}x+v$
		\item[Observer:] $d\hat{\mathbf{x}}=\mathbf{A}\hat{\mathbf{x}}+\mathbf{B}u+\mathbf{L}(y-\mathbf{C}\hat{\mathbf{x}})$
		\item[Error $\mathbf{e}=\mathbf{x}-\hat{\mathbf{x}}$:] $d\mathbf{e}/dt=(\mathbf{A}-\mathbf{LC})\mathbf{e}+\mathbf{B}_1 w - \mathbf{L}v.$
	\end{description}
\end{itemize}

\endinput

%%% Local Variables: 
%%% mode: latex
%%% TeX-master: "notes"
%%% End:
\end{slide}
\fi

Taking Laplace transforms, ignoring ICs:
The last example had a system TF with no zeros. In this case it is easy to construct the equivalent classical controller. We had the feedback law:
\[
u=r-5x_1-156x_2
\]
so, taking Laplace transforms:
\[
U(s) = R(s) - 5X_1(s) - 156X_2(s)
\]

Now $y=7x_2$ and $\dot{x}_2=x_1$ therefore $X_2(s)=Y(s)/7$ and $X_1(s)=sX_2(s)=sY(s)/7$. Therefore
\[
	U(s) =  R(s) - \frac{1}{7}(5s+156)Y(s)
\]

\endinput

%%% Local Variables: 
%%% mode: latex
%%% TeX-master: "notes"
%%% End:
\ifslidesonly
\begin{slide}
	\heading{Taking Laplace Transforms}
   The last example had a system TF with no zeros. In this case it is easy to construct the equivalent classical controller. We had the feedback law:
\[
u=r-5x_1-156x_2
\]
so, taking Laplace transforms:
\[
U(s) = R(s) - 5X_1(s) - 156X_2(s)
\]

Now $y=7x_2$ and $\dot{x}_2=x_1$ therefore $X_2(s)=Y(s)/7$ and $X_1(s)=sX_2(s)=sY(s)/7$. Therefore
\[
	U(s) =  R(s) - \frac{1}{7}(5s+156)Y(s)
\]

\endinput

%%% Local Variables: 
%%% mode: latex
%%% TeX-master: "notes"
%%% End:
\end{slide}
\fi


Alternatively, since
\begin{center}
	\resizebox{300pt}{!}{\includegraphics{pictures/tfmodel.pdf}}
\end{center}

\textbf{Note}: the DC gain is affected -- this could be compensated for by introducing a gain term in series with input $R$.

\endinput

%%% Local Variables: 
%%% mode: latex
%%% TeX-master: "notes"
%%% End:
\ifslidesonly
\begin{slide}
	\heading{Alternative Formulation (1)}
   \begin{center}
	\resizebox{300pt}{!}{\includegraphics{pictures/tfmodel.pdf}}
\end{center}

\textbf{Note}: the DC gain is affected -- this could be compensated for by introducing a gain term in series with input $R$.

\endinput

%%% Local Variables: 
%%% mode: latex
%%% TeX-master: "notes"
%%% End:
\end{slide}
\fi
Now
\[
U(s) =  - {\bf{K\hat X}}(s) 
\]
so
\[
U (s) =   - {\bf{K}}\left( {{\bf{M}}^{ - 1} {\bf{B}}U(s) + {\bf{M}}^{ - 1} {\bf{L}}Y(s)} \right) 
\]
\[
 \left( {1 + {\bf{KM}}^{ - 1} {\bf{B}}} \right)U(s) =  - {\bf{KM}}^{ - 1} {\bf{L}}Y(s)
\]

\[ 
H\left( s \right) =  - \frac{U(s)}{Y(sßß)} = \frac{{{\bf{KM}}^{ - 1} {\bf{L}}}}{{1 + {\bf{KM}}^{ - 1} {\bf{B}}}} 
\]




\endinput

%%% Local Variables: 
%%% mode: latex
%%% TeX-master: "notes"
%%% End:
\ifslidesonly
\begin{slide}
	\heading{Alternative Formulation (2)}
   Now
\[
U(s) =  - {\bf{K\hat X}}(s) 
\]
so
\[
U (s) =   - {\bf{K}}\left( {{\bf{M}}^{ - 1} {\bf{B}}U(s) + {\bf{M}}^{ - 1} {\bf{L}}Y(s)} \right) 
\]
\[
 \left( {1 + {\bf{KM}}^{ - 1} {\bf{B}}} \right)U(s) =  - {\bf{KM}}^{ - 1} {\bf{L}}Y(s)
\]

\[ 
H\left( s \right) =  - \frac{U(s)}{Y(sßß)} = \frac{{{\bf{KM}}^{ - 1} {\bf{L}}}}{{1 + {\bf{KM}}^{ - 1} {\bf{B}}}} 
\]




\endinput

%%% Local Variables: 
%%% mode: latex
%%% TeX-master: "notes"
%%% End:
\end{slide}
\fi

% section the_equivalent_classical_compensator (end)



 
\section*{Introducing the Reference Input} % (fold)
\label{sec:introducing_the_reference_input}

\ifslidesonly
\begin{slide}
   \heading{Introducing the Reference Input}
Two cases considered:
\begin{enumerate}
	\item The Normal Position
	\begin{itemize}
		\item The reference input is introduced as $u=r-\mathbf{K}\hat{\mathbf{x}}$.
	\end{itemize}
	\item Observer driven by the tracking error
	\begin{itemize}
		\item Sense the tracking error and use this to control the system.
		\item The tracking error is the difference between the desired and actual outputs $\tilde e =  - r + y$
	\end{itemize}
	Systems in each case have different properties.
\end{enumerate}
   
\end{slide}
\fi
\subsection*{1. The Normal Position} % (fold)
\label{sub:1_the_normal_position}


\begin{center}
	\begin{tabular}{|c|c|c|}
	\hline
	\textbf{Sub-system} & \textbf{Controllable?} & \textbf{Observable?}\\
	\hline
	$\mathbf{S}_1$ & Yes & No\\
	\hline
	$\mathbf{S}_2$ & Yes & Yes\\
	\hline
	$\mathbf{S}_3$ & No & Yes\\
	\hline
	$\mathbf{S}_4$ & No & No\\
	\hline
	\end{tabular}	
\end{center}

\endinput

%%% Local Variables: 
%%% mode: latex
%%% TeX-master: "notes"
%%% End:
\ifslidesonly
\begin{slide}
	\heading{1. The Normal Position}
   
\begin{center}
	\begin{tabular}{|c|c|c|}
	\hline
	\textbf{Sub-system} & \textbf{Controllable?} & \textbf{Observable?}\\
	\hline
	$\mathbf{S}_1$ & Yes & No\\
	\hline
	$\mathbf{S}_2$ & Yes & Yes\\
	\hline
	$\mathbf{S}_3$ & No & Yes\\
	\hline
	$\mathbf{S}_4$ & No & No\\
	\hline
	\end{tabular}	
\end{center}

\endinput

%%% Local Variables: 
%%% mode: latex
%%% TeX-master: "notes"
%%% End:
\end{slide}
\fi



\subsubsection*{Finding $F(s)$ and $H(s)$} % (fold)
\label{ssub:finding_f_s_and_h_s_}

For the observer we have:
The full system response for the state-space model is simply the
sum of the zero-state and zero-input responses:
\begin{eqnarray*}\mathbf{Y}_{\mathrm{full}}(s) &=& \mathbf{Y}_{\mathrm{zs}}(s) +
\mathbf{Y}_{\mathrm{zi}}(s)\\ &=&
\mathbf{C}\Phis{}\left[\mathbf{x}(0)+\mathbf{B}\mathbf{U}(s)\right] + \mathbf{DU}(s).\end{eqnarray*}
\endinput

\ifslidesonly
\begin{slide}
	\heading{Finding$F(s)$ and $H(s)$: Observer}
   The full system response for the state-space model is simply the
sum of the zero-state and zero-input responses:
\begin{eqnarray*}\mathbf{Y}_{\mathrm{full}}(s) &=& \mathbf{Y}_{\mathrm{zs}}(s) +
\mathbf{Y}_{\mathrm{zi}}(s)\\ &=&
\mathbf{C}\Phis{}\left[\mathbf{x}(0)+\mathbf{B}\mathbf{U}(s)\right] + \mathbf{DU}(s).\end{eqnarray*}
\endinput

\end{slide}
\fi

For the controller we have:
\begin{itemize}
	\item Ackermann's formula is useful for SISO systems of order  $n\le 10$.
	\item $\mathcal{C}$ becomes numerically inaccurate for large $n$.
	\item The system must be \emph{controllable} for $\mathcal{C}^{-1}$ to exist.
\end{itemize}

\endinput

%%% Local Variables: 
%%% mode: latex
%%% TeX-master: "notes"
%%% End:
\ifslidesonly
\begin{slide}
	\heading{Finding$F(s)$ and $H(s)$: Controller}
   \begin{itemize}
	\item Ackermann's formula is useful for SISO systems of order  $n\le 10$.
	\item $\mathcal{C}$ becomes numerically inaccurate for large $n$.
	\item The system must be \emph{controllable} for $\mathcal{C}^{-1}$ to exist.
\end{itemize}

\endinput

%%% Local Variables: 
%%% mode: latex
%%% TeX-master: "notes"
%%% End:
\end{slide}
\fi

Therefore, for the combined observer-controller
From Matlab CST, \verb|help acker|:
\verb|K = ACKER(A,B,P)|  calculates the feedback gain matrix $\mathbf{K}$ such that
the single input system
\[
\dot{\mathbf{x}}=\mathbf{Ax}+\mathbf{B}u
\]
with a feedback law of  $u = -\mathbf{Kx}$  has closed loop poles at the 
values specified in vector $\mathbf{P}$, i.e.,  \texttt{P = eig(A-B*K)}.

\textbf{Note}: This algorithm uses Ackermann's formula.  This method
is NOT numerically reliable and starts to break down rapidly
for problems of order greater than 10, or for weakly controllable
systems.  A warning message is printed if the nonzero closed-loop
poles are greater than 10\% from the desired locations specified 
in $\mathbf{P}$.


\endinput

%%% Local Variables: 
%%% mode: latex
%%% TeX-master: "notes"
%%% End:
\ifslidesonly
\begin{slide}
	\heading{Finding$F(s)$ and $H(s)$: Combined Observer-Controller}
   From Matlab CST, \verb|help acker|:
\verb|K = ACKER(A,B,P)|  calculates the feedback gain matrix $\mathbf{K}$ such that
the single input system
\[
\dot{\mathbf{x}}=\mathbf{Ax}+\mathbf{B}u
\]
with a feedback law of  $u = -\mathbf{Kx}$  has closed loop poles at the 
values specified in vector $\mathbf{P}$, i.e.,  \texttt{P = eig(A-B*K)}.

\textbf{Note}: This algorithm uses Ackermann's formula.  This method
is NOT numerically reliable and starts to break down rapidly
for problems of order greater than 10, or for weakly controllable
systems.  A warning message is printed if the nonzero closed-loop
poles are greater than 10\% from the desired locations specified 
in $\mathbf{P}$.


\endinput

%%% Local Variables: 
%%% mode: latex
%%% TeX-master: "notes"
%%% End:
\end{slide}
\fi


Comparing with: $$U=F(s)R-H(s)Y$$
\[
F(s) = \frac{1}{\mathbf{KM}^{-1}\mathbf{B}+1}
\]
and
\[
H(s) = \frac{\mathbf{KM}^{-1}\mathbf{L}}{\mathbf{KM}^{-1}\mathbf{B}+1}
\]

% subsubsection finding_f_s_and_h_s_ (end)

 

\subsubsection*{A useful theorem} % (fold)
\label{ssub:a_useful_theorem}

If $\mathbf{M}$ is square and $\mathbf{V}$ is a row vector and $\mathbf{W}$ is a column vector then,
\[
\mathbf{VM}^{-1}\mathbf{W}=\frac{\det\left(\mathbf{M}+\mathbf{WV}\right)}{\det{\mathbf{M}}}-1
\]
 


\endinput

%%% Local Variables: 
%%% mode: latex
%%% TeX-master: "notes"
%%% End:
\ifslidesonly
\begin{slide}
	\heading{A Useful Theorem}
   If $\mathbf{M}$ is square and $\mathbf{V}$ is a row vector and $\mathbf{W}$ is a column vector then,
\[
\mathbf{VM}^{-1}\mathbf{W}=\frac{\det\left(\mathbf{M}+\mathbf{WV}\right)}{\det{\mathbf{M}}}-1
\]
 


\endinput

%%% Local Variables: 
%%% mode: latex
%%% TeX-master: "notes"
%%% End:
\end{slide}
\fi

 
% subsubsection a_useful_theorem (end)
\subsubsection*{Zeros and Poles of $F(s)$ and $H(s)$} % (fold)
\label{ssub:zeros_and_poles_of_f_s_and_h_s_}

By analogy with previous work (see Section `\emph{Some Important Properties}', in Lecture 15), the TF from reference input $r$  to output $y$  is:
\[
\frac{{Y(s)}}{{R(s)}} = \frac{{\det \left[ {\begin{array}{*{20}c}
   {(s{\bf{I}} - {\bf{A}} + {\bf{BK}})} & { - {\bf{B}}}  \\
   {({\bf{C}} - {\bf{DK}})} & {\bf{D}}  \\
\end{array}} \right]}}{{\det (s{\bf{I}} - {\bf{A}} + {\bf{BK}})}}
\]

The closed loop TF zeros are determined by the numerator determinant.


\endinput

%%% Local Variables: 
%%% mode: latex
%%% TeX-master: "notes"
%%% End:
\ifslidesonly
\begin{slide}
	\heading{Zeros and Poles of $F(s)$ (1)}
   By analogy with previous work (see Section `\emph{Some Important Properties}', in Lecture 15), the TF from reference input $r$  to output $y$  is:
\[
\frac{{Y(s)}}{{R(s)}} = \frac{{\det \left[ {\begin{array}{*{20}c}
   {(s{\bf{I}} - {\bf{A}} + {\bf{BK}})} & { - {\bf{B}}}  \\
   {({\bf{C}} - {\bf{DK}})} & {\bf{D}}  \\
\end{array}} \right]}}{{\det (s{\bf{I}} - {\bf{A}} + {\bf{BK}})}}
\]

The closed loop TF zeros are determined by the numerator determinant.


\endinput

%%% Local Variables: 
%%% mode: latex
%%% TeX-master: "notes"
%%% End:
\end{slide}
\fi

Adding $\mathbf{K}$ times the $2^\mathrm{nd}$ column to the first cancels terms whilst leaving the determinant unchanged.

The new form for the TF is:
\[
\frac{{Y(s)}}{{R(s)}} = \frac{{\det \left[ {\begin{array}{*{20}c}
   {(s{\bf{I}} - {\bf{A}})} & { - {\bf{B}}}  \\
   {\bf{C}} & {\bf{D}}  \\
\end{array}} \right]}}{{\det (s{\bf{I}} - {\bf{A}} + {\bf{BK}})}}
\]

Notice now that numerator is identical to that of the open loop TF. This implies that the state feedback control has left the open loop zeros unchanged. The different denominator is due to the feedback action which alters the pole positions as required.

\endinput

%%% Local Variables: 
%%% mode: latex
%%% TeX-master: "notes"
%%% End:
\ifslidesonly
\begin{slide}
	\heading{Zeros and Poles of $F(s)$ (2)}
   Adding $\mathbf{K}$ times the $2^\mathrm{nd}$ column to the first cancels terms whilst leaving the determinant unchanged.

The new form for the TF is:
\[
\frac{{Y(s)}}{{R(s)}} = \frac{{\det \left[ {\begin{array}{*{20}c}
   {(s{\bf{I}} - {\bf{A}})} & { - {\bf{B}}}  \\
   {\bf{C}} & {\bf{D}}  \\
\end{array}} \right]}}{{\det (s{\bf{I}} - {\bf{A}} + {\bf{BK}})}}
\]

Notice now that numerator is identical to that of the open loop TF. This implies that the state feedback control has left the open loop zeros unchanged. The different denominator is due to the feedback action which alters the pole positions as required.

\endinput

%%% Local Variables: 
%%% mode: latex
%%% TeX-master: "notes"
%%% End:
\end{slide}
\fi

Similarly for $H(s)$:
When a zero is close to a pole in the TF there is a marked increase in the feedback gains. This effect is best illustrated with an example.

Given a system with a TF,
\[
\frac{Y(s)}{U(s)}=\frac{s-z}{s-p}=1+\frac{p-z}{s-p}
\]
find the control law to move the pole to $p_c$.



\endinput

%%% Local Variables: 
%%% mode: latex
%%% TeX-master: "notes"
%%% End:
\ifslidesonly
\begin{slide}
	\heading{Zeros and Poles of $H(s)$}
   When a zero is close to a pole in the TF there is a marked increase in the feedback gains. This effect is best illustrated with an example.

Given a system with a TF,
\[
\frac{Y(s)}{U(s)}=\frac{s-z}{s-p}=1+\frac{p-z}{s-p}
\]
find the control law to move the pole to $p_c$.



\endinput

%%% Local Variables: 
%%% mode: latex
%%% TeX-master: "notes"
%%% End:
\end{slide}
\fi


% subsubsection zeros_and_poles_of_f_s_and_h_s_ (end)


\subsubsection*{The Overall Closed Loop TF} % (fold)
\label{ssub:the_overall_closed_loop_tf}

Using the observer canonical form,
\[
\dot{x}=px+(p - z)u,\ y = x + u.
\]

Design the feedback to move the closed-loop pole to $p_c$.
Now,
\[
\mathbf{A}=p;\ \mathbf{B}=p-z;\ \mathbf{K}=k_1.
\]
Desired CE polynomial: $\alpha_c(s)=s-p_c$. Actual CE polynomial: 
$\det(s\mathbf{I}-\mathbf{A}-\mathbf{BK}) = s - p + (p - z)k_1.$

Comparing the constant:
\begin{eqnarray*}
	-p_c & = & -p + (p - z)k_1 \\
	k_1 & = & \frac{p-p_c}{p-z}
\end{eqnarray*}


\endinput

%%% Local Variables: 
%%% mode: latex
%%% TeX-master: "notes"
%%% End:
\ifslidesonly
\begin{slide}
	\heading{The Overall Closed Loop TF (1)}
   Using the observer canonical form,
\[
\dot{x}=px+(p - z)u,\ y = x + u.
\]

Design the feedback to move the closed-loop pole to $p_c$.
Now,
\[
\mathbf{A}=p;\ \mathbf{B}=p-z;\ \mathbf{K}=k_1.
\]
Desired CE polynomial: $\alpha_c(s)=s-p_c$. Actual CE polynomial: 
$\det(s\mathbf{I}-\mathbf{A}-\mathbf{BK}) = s - p + (p - z)k_1.$

Comparing the constant:
\begin{eqnarray*}
	-p_c & = & -p + (p - z)k_1 \\
	k_1 & = & \frac{p-p_c}{p-z}
\end{eqnarray*}


\endinput

%%% Local Variables: 
%%% mode: latex
%%% TeX-master: "notes"
%%% End:
\end{slide}
\fi

We know from previous work that the denominator, corresponding to the closed loop CE of the overall system must, from the separation principle, be equivalent to:
\[
\alpha_e(s)\alpha_c(s)
\]
therefore,
\[
\frac{{Y(s)}}{{R(s)}} = \frac{{\alpha _e (s)\alpha _z (s)}}{{\alpha _e (s)\alpha _c (s)}} = \frac{{\alpha _z (s)}}{{\alpha _c (s)}}
\]



\endinput

%%% Local Variables: 
%%% mode: latex
%%% TeX-master: "notes"
%%% End:
\ifslidesonly
\begin{slide}
	\heading{The Overall Closed Loop TF (2)}
   We know from previous work that the denominator, corresponding to the closed loop CE of the overall system must, from the separation principle, be equivalent to:
\[
\alpha_e(s)\alpha_c(s)
\]
therefore,
\[
\frac{{Y(s)}}{{R(s)}} = \frac{{\alpha _e (s)\alpha _z (s)}}{{\alpha _e (s)\alpha _c (s)}} = \frac{{\alpha _z (s)}}{{\alpha _c (s)}}
\]



\endinput

%%% Local Variables: 
%%% mode: latex
%%% TeX-master: "notes"
%%% End:
\end{slide}
\fi

Large feedback control gains are required if:
\begin{enumerate}
	\item There exist almost cancelling pole-zero pairs in the open loop TF, making the system almost uncontrollable.
\begin{center}
	\resizebox{300pt}{!}{\includegraphics{pictures/zeroloc.pdf}}
\end{center}
	(Notice that in this condition the input $u$ is almost disconnected from the integrator for the state.)
	\item One tries to move the poles a long way, ($|p-p_c|$ large). 
	
	This imposes a practical limit on how arbitrarily 
	the poles can be placed. You cannot make a slow system fast without using large gains requiring powerful, 
	expensive actuators to force the plant response. Indeed, excessively large forces may destroy the plant.
\end{enumerate}

\endinput

%%% Local Variables: 
%%% mode: latex
%%% TeX-master: "notes"
%%% End:
\ifslidesonly
\begin{slide}
	\heading{The Overall Closed Loop TF: Comments}
   Large feedback control gains are required if:
\begin{enumerate}
	\item There exist almost cancelling pole-zero pairs in the open loop TF, making the system almost uncontrollable.
\begin{center}
	\resizebox{300pt}{!}{\includegraphics{pictures/zeroloc.pdf}}
\end{center}
	(Notice that in this condition the input $u$ is almost disconnected from the integrator for the state.)
	\item One tries to move the poles a long way, ($|p-p_c|$ large). 
	
	This imposes a practical limit on how arbitrarily 
	the poles can be placed. You cannot make a slow system fast without using large gains requiring powerful, 
	expensive actuators to force the plant response. Indeed, excessively large forces may destroy the plant.
\end{enumerate}

\endinput

%%% Local Variables: 
%%% mode: latex
%%% TeX-master: "notes"
%%% End:
\end{slide}
\fi



% subsubsection the_overall_closed_loop_tf (end)

% subsection 1_the_normal_position (end)


 
\subsection*{2. Observer Driven by the Tracking Error
} % (fold)
\label{sub:2_observer_driven_by_the_tracking_error_}

Sometimes it is desired to sense the tracking error and use this to control the system.

The tracking error is the difference between the desired and actual outputs,
\[
\tilde e =  - r + y
\]

 
Some sensors can only measure a difference between two measurands. eg  a thermocouple can only sense the temperature difference between its hot and cold junctions.
\begin{center}
	\resizebox{!}{2in}{\includegraphics{pictures/fig3.pdf}}
\end{center}
\ifslidesonly
\begin{slide}
   \heading{2. Observer Driven by the Tracking Error}
The tracking error is the difference between the desired and actual outputs,
\[
\tilde e =  - r + y
\]
   \begin{center}
	\resizebox{200pt}{!}{\includegraphics{pictures/fig3.pdf}}
   \end{center}
\end{slide}
\fi
Redraw:
\begin{center}
	\resizebox{250pt}{!}{\includegraphics{pictures/fig4.pdf}}
\end{center}
\ifslidesonly
\begin{slide}
   \heading{Redraw Block Diagram}
   \begin{center}
	\resizebox{250pt}{!}{\includegraphics{pictures/fig4.pdf}}
\end{center}
\end{slide}
\fi

In this configuration the observer is driven by:
\[
\tilde E =  Y + R
\] 
Hence the observer dynamics are:
\[
s\hat{\mathbf{X}}=\mathbf{A}\hat{\mathbf{X}}+\mathbf{B}U+\mathbf{L}(\tilde E - \mathbf{C}\hat{\mathbf{X}})
\]
therefore,
\begin{eqnarray*}
	\underbrace {\left( {s{\bf{I}} - {\bf{A}} + {\bf{LC}}} \right)}_{\bf{M}}{\bf{\hat X}} & = & {\bf{B}}U + {\bf{L}}\tilde E \\
	{\bf{\hat X}} & = & {\bf{M}}^{ - 1} {\bf{B}}U + {\bf{M}}^{ - 1} {\bf{L}}\tilde E
\end{eqnarray*}



\endinput

%%% Local Variables: 
%%% mode: latex
%%% TeX-master: "notes"
%%% End:
\ifslidesonly
\begin{slide}
	\heading{Observer Driven by the Tracking Error (1)}
   In this configuration the observer is driven by:
\[
\tilde E =  Y + R
\] 
Hence the observer dynamics are:
\[
s\hat{\mathbf{X}}=\mathbf{A}\hat{\mathbf{X}}+\mathbf{B}U+\mathbf{L}(\tilde E - \mathbf{C}\hat{\mathbf{X}})
\]
therefore,
\begin{eqnarray*}
	\underbrace {\left( {s{\bf{I}} - {\bf{A}} + {\bf{LC}}} \right)}_{\bf{M}}{\bf{\hat X}} & = & {\bf{B}}U + {\bf{L}}\tilde E \\
	{\bf{\hat X}} & = & {\bf{M}}^{ - 1} {\bf{B}}U + {\bf{M}}^{ - 1} {\bf{L}}\tilde E
\end{eqnarray*}



\endinput

%%% Local Variables: 
%%% mode: latex
%%% TeX-master: "notes"
%%% End:
\end{slide}
\fi

Now $U=-\mathbf{K}\hat{\mathbf{X}}$, therefore
\begin{eqnarray*}
	U & = &  - \bf{KM}^{ - 1} {\bf{B}}U - {\bf{KM}}^{ - 1} {\bf{L}}\tilde E \\
	U & = &  - \frac{{{\bf{KM}}^{ - 1} {\bf{L}}}}{{1 + {\bf{KM}}^{ - 1} {\bf{B}}}}\tilde E \\
	H\left( s \right) & = &  - \frac{U}{{\tilde E}} = \frac{{{\bf{KM}}^{ - 1} {\bf{L}}}}{{1 + {\bf{KM}}^{ - 1} {\bf{B}}}}
\end{eqnarray*}


\endinput

%%% Local Variables: 
%%% mode: latex
%%% TeX-master: "notes"
%%% End:
\ifslidesonly
\begin{slide}
	\heading{Observer Driven by the Tracking Error (2)}
   Now $U=-\mathbf{K}\hat{\mathbf{X}}$, therefore
\begin{eqnarray*}
	U & = &  - \bf{KM}^{ - 1} {\bf{B}}U - {\bf{KM}}^{ - 1} {\bf{L}}\tilde E \\
	U & = &  - \frac{{{\bf{KM}}^{ - 1} {\bf{L}}}}{{1 + {\bf{KM}}^{ - 1} {\bf{B}}}}\tilde E \\
	H\left( s \right) & = &  - \frac{U}{{\tilde E}} = \frac{{{\bf{KM}}^{ - 1} {\bf{L}}}}{{1 + {\bf{KM}}^{ - 1} {\bf{B}}}}
\end{eqnarray*}


\endinput

%%% Local Variables: 
%%% mode: latex
%%% TeX-master: "notes"
%%% End:
\end{slide}
\fi

A most effective technique is to use optimal control to achieve a compromise between control effort, $u$, and error, $e$.
i.e. Find the feedback vector $\mathbf{K}$ such as to minimise,
\[
J=\int_0^{\infty}\left(e^2+\frac{u^2}{k}\right) dt
\]
where the choice of the parameter $k$ determines the required compromise between,
\begin{itemize}
	\item High Accuracy for High Control Effort (use a large value for $k$)
	\item Lower Accuracy for Reduced Control Effort (use a smaller value for $k$)
\end{itemize}

\endinput

%%% Local Variables: 
%%% mode: latex
%%% TeX-master: "notes"
%%% End:
\ifslidesonly
\begin{slide}
	\heading{Observer Driven by the Tracking Error: $H(s)$}
   A most effective technique is to use optimal control to achieve a compromise between control effort, $u$, and error, $e$.
i.e. Find the feedback vector $\mathbf{K}$ such as to minimise,
\[
J=\int_0^{\infty}\left(e^2+\frac{u^2}{k}\right) dt
\]
where the choice of the parameter $k$ determines the required compromise between,
\begin{itemize}
	\item High Accuracy for High Control Effort (use a large value for $k$)
	\item Lower Accuracy for Reduced Control Effort (use a smaller value for $k$)
\end{itemize}

\endinput

%%% Local Variables: 
%%% mode: latex
%%% TeX-master: "notes"
%%% End:
\end{slide}
\fi

The overall TF is:

\begin{eqnarray*}
	\frac{Y(s)}{R(s)} &=& \frac{G(s)}{1+G(s)H(s)}= \frac{\frac{\alpha_z(s)}{\alpha_p(s)}\times\frac{\alpha_2(s)}{\alpha_1(s)}}{1+\frac{\alpha_z(s)}{\alpha_p(s)}\times\frac{\alpha_2(s)}{\alpha_1(s)}} \\
	\frac{Y(s)}{R(s)} &=& \frac{\alpha_z(s)\alpha_2(s)}{\alpha_p(s)\alpha_1(s)+\alpha_z(s)\alpha_2(s)}=\frac{\alpha_z\alpha_2}{\alpha_c\alpha_e}
\end{eqnarray*}
 



\endinput

%%% Local Variables: 
%%% mode: latex
%%% TeX-master: "notes"
%%% End:
\ifslidesonly
\begin{slide}
	\heading{Observer Driven by the Tracking Error: Overall TF}
   
\begin{eqnarray*}
	\frac{Y(s)}{R(s)} &=& \frac{G(s)}{1+G(s)H(s)}= \frac{\frac{\alpha_z(s)}{\alpha_p(s)}\times\frac{\alpha_2(s)}{\alpha_1(s)}}{1+\frac{\alpha_z(s)}{\alpha_p(s)}\times\frac{\alpha_2(s)}{\alpha_1(s)}} \\
	\frac{Y(s)}{R(s)} &=& \frac{\alpha_z(s)\alpha_2(s)}{\alpha_p(s)\alpha_1(s)+\alpha_z(s)\alpha_2(s)}=\frac{\alpha_z\alpha_2}{\alpha_c\alpha_e}
\end{eqnarray*}
 



\endinput

%%% Local Variables: 
%%% mode: latex
%%% TeX-master: "notes"
%%% End:
\end{slide}
\fi

\begin{itemize}
	\item In this case we see that the overall TF contains the poles of the observer as well as the controller.
	\item Whereas in the normal position changes in the reference input do not excite the error dynamics of the observer, in this configuration they do.
	\item As a result the difference between the observer and the plant states is affected during operation and take further time to settle down.
\end{itemize}

\endinput

%%% Local Variables: 
%%% mode: latex
%%% TeX-master: "notes"
%%% End:
\ifslidesonly
\begin{slide}
	\heading{Observer Driven by the Tracking Error: Comments}
   \begin{itemize}
	\item In this case we see that the overall TF contains the poles of the observer as well as the controller.
	\item Whereas in the normal position changes in the reference input do not excite the error dynamics of the observer, in this configuration they do.
	\item As a result the difference between the observer and the plant states is affected during operation and take further time to settle down.
\end{itemize}

\endinput

%%% Local Variables: 
%%% mode: latex
%%% TeX-master: "notes"
%%% End:
\end{slide}
\fi


% subsection 2_observer_driven_by_the_tracking_error_ (end)

\ifslidesonly
\begin{slide}
	\heading{Summary of this Lecture}
   \begin{itemize}
   	\item Dynamics of the combined controller-observer system
\begin{itemize}
	\item Separation principle
	\item Equivalent classical controller
\end{itemize}
\item Introducing the reference input
\begin{enumerate}
	\item The normal position
	\item Observer driven by the tracking error
\end{enumerate}
   \end{itemize}
\end{slide}
\fi


% section introducing_the_reference_input (end)




%----------------------------------------------------------------
% The end of notes
% ----------------------------------------------------------------
\endinput

%%% Local Variables: 
%%% mode: latex
%%% TeX-master: t
%%% End: 

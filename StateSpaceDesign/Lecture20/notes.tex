%% State Space Modelling of Dynamic Systems
%% Lecture 20: State Feedback Control
\def\FileDate{10/02/02}
\def\FileVersion{1.0}
% ----------------------------------------------------------------
% Notes pages *********************************************************
% ----------------------------------------------------------------

\begin{slide}
   \heading{State Feedback Control}
One of the advantages of state space models is that it is possible to apply state feedback to place the closed loop poles into any desired positions.

\textbf{State Space Design Methodology}

\begin{enumerate}
	\item Design control law to place closed loop poles where desired
	\item If full state not available for feedback, then design an \emph{Observer} to compute the states from the system output
	\item Combine \emph{Observer} and \emph{Controller} -- this takes the place of the \emph{Classical Compensator}
	\item Introduce the \emph{Reference Input} -- affects the closed loop zeros but not the poles making it possible to improve the transient response and tracking accuracy
\end{enumerate}	
\end{slide}


\begin{slide}
   \heading{State Feedback Compensator}
   \begin{center}
   	\resizebox{280pt}{!}{\includegraphics{pictures/statefb.pdf}}
   \end{center}
\end{slide}

\ifslidesonly
\begin{slide}
   \heading{This Lecture}
   \begin{itemize}
   	\item Finding the control law
   	\item State feedback for controller canonical form
   	\item Transfer function model
   	\item Ackermann's formula
   	\item Effect of state feedback on closed-loop zeros
   	\item Effect of plant zeros on the feedback gains
   	\item Pole-selection for good design
   \end{itemize}
\end{slide}
\fi
\section*{Finding the Control Law} % (fold)
\label{sec:finding_the_control_law}

We shall only consider SISO systems here.

Therefore the dynamics of the combined system are:
\[
\left[ {\begin{array}{*{20}c}
   {{\bf{\dot x}}}  \\
   {{\bf{\dot e}}}  \\
\end{array}} \right] = \left[ {\begin{array}{*{20}c}
   {\left( {{\bf{A}} - {\bf{BK}}} \right)} & {{\bf{BK}}}  \\
   {\bf{0}} & {\left( {{\bf{A}} - {\bf{LC}}} \right)}  \\
\end{array}} \right]\left[ {\begin{array}{*{20}c}
   {\bf{x}}  \\
   {\bf{e}}  \\
\end{array}} \right] + \left[ {\begin{array}{*{20}c}
   {\bf{B}}  \\
   {\bf{0}}  \\
	\end{array}} \right]r
\]

\endinput

%%% Local Variables: 
%%% mode: latex
%%% TeX-master: "notes"
%%% End:
\ifslidesonly
\begin{slide}
   \heading{Finding the Control Law (1)}
   Therefore the dynamics of the combined system are:
\[
\left[ {\begin{array}{*{20}c}
   {{\bf{\dot x}}}  \\
   {{\bf{\dot e}}}  \\
\end{array}} \right] = \left[ {\begin{array}{*{20}c}
   {\left( {{\bf{A}} - {\bf{BK}}} \right)} & {{\bf{BK}}}  \\
   {\bf{0}} & {\left( {{\bf{A}} - {\bf{LC}}} \right)}  \\
\end{array}} \right]\left[ {\begin{array}{*{20}c}
   {\bf{x}}  \\
   {\bf{e}}  \\
\end{array}} \right] + \left[ {\begin{array}{*{20}c}
   {\bf{B}}  \\
   {\bf{0}}  \\
	\end{array}} \right]r
\]

\endinput

%%% Local Variables: 
%%% mode: latex
%%% TeX-master: "notes"
%%% End:
\end{slide}
\fi


\begin{center}
	\begin{tabular}{|c|c|c|}
	\hline
	\textbf{Sub-system} & \textbf{Controllable?} & \textbf{Observable?}\\
	\hline
	$\mathbf{S}_1$ & Yes & No\\
	\hline
	$\mathbf{S}_2$ & Yes & Yes\\
	\hline
	$\mathbf{S}_3$ & No & Yes\\
	\hline
	$\mathbf{S}_4$ & No & No\\
	\hline
	\end{tabular}	
\end{center}

\endinput

%%% Local Variables: 
%%% mode: latex
%%% TeX-master: "notes"
%%% End:
\ifslidesonly
\begin{slide}
	\heading{Finding the Control Law (2)}
   
\begin{center}
	\begin{tabular}{|c|c|c|}
	\hline
	\textbf{Sub-system} & \textbf{Controllable?} & \textbf{Observable?}\\
	\hline
	$\mathbf{S}_1$ & Yes & No\\
	\hline
	$\mathbf{S}_2$ & Yes & Yes\\
	\hline
	$\mathbf{S}_3$ & No & Yes\\
	\hline
	$\mathbf{S}_4$ & No & No\\
	\hline
	\end{tabular}	
\end{center}

\endinput

%%% Local Variables: 
%%% mode: latex
%%% TeX-master: "notes"
%%% End:
\end{slide}
\fi

 
\begin{itemize}
	\item We shall determine the classical compensator TF which is equivalent to the combined observer-controller. 
	\item This is simpler to do if we remove the reference input, for the time being.
\end{itemize}
\begin{center}
	\resizebox{300pt}{!}{\includegraphics{pictures/fig1.pdf}}
\end{center}
\endinput

%%% Local Variables: 
%%% mode: latex
%%% TeX-master: "notes"
%%% End:
\ifslidesonly
\begin{slide}
	\heading{Finding the Control Law (3)}
   \begin{itemize}
	\item We shall determine the classical compensator TF which is equivalent to the combined observer-controller. 
	\item This is simpler to do if we remove the reference input, for the time being.
\end{itemize}
\begin{center}
	\resizebox{300pt}{!}{\includegraphics{pictures/fig1.pdf}}
\end{center}
\endinput

%%% Local Variables: 
%%% mode: latex
%%% TeX-master: "notes"
%%% End:
\end{slide}
\fi

In particular, the solution of:
\[
\frac{d\mathbf{x}}{dt}=\mathbf{Ax}
\]
given $\mathbf{x}=\mathbf{x}_0$ at $t=0$ is:
\[
\mathbf{x}=e^{\mathbf{A}t}\mathbf{x}_0
\]
               
The term  $\mathbf{\phi}(t) = e^{\mathbf{A}t}$  is known as the state transition matrix because it shows how time, $t$, transforms the initial state vector into the present one.

\endinput

%%% Local Variables: 
%%% mode: latex
%%% TeX-master: "notes"
%%% End:
\ifslidesonly
\begin{slide}
	\heading{Finding the Control Law (4)}
   In particular, the solution of:
\[
\frac{d\mathbf{x}}{dt}=\mathbf{Ax}
\]
given $\mathbf{x}=\mathbf{x}_0$ at $t=0$ is:
\[
\mathbf{x}=e^{\mathbf{A}t}\mathbf{x}_0
\]
               
The term  $\mathbf{\phi}(t) = e^{\mathbf{A}t}$  is known as the state transition matrix because it shows how time, $t$, transforms the initial state vector into the present one.

\endinput

%%% Local Variables: 
%%% mode: latex
%%% TeX-master: "notes"
%%% End:
\end{slide}
\fi



\subsection*{Example 1} % (fold)
\label{sub:example_1}

\textbf{Problem}: Given,
\[
{\bf{\dot x}} = \left[ {\begin{array}{*{20}c}
   { - 4} & 0  \\
   0 & { - 11}  \\
\end{array}} \right]{\bf{x}} + \left[ {\begin{array}{*{20}c}
   1  \\
   { - 1}  \\
\end{array}} \right]u
\]
find the feedback control law which places the closed-loop poles at: $-10\pm j10$.

\textbf{SOLUTION}:
\begin{eqnarray*}
	0 & = & \det \left[ {s{\bf{I}} - {\bf{A}} + {\bf{BK}}} \right] = \det \left\{ {\left. {\left[ {\begin{array}{*{20}c}
	   {s + 4} & 0  \\
	   0 & {s + 11}  \\
	\end{array}} \right] + \left[ {\begin{array}{*{20}c}
	   1  \\
	   { - 1}  \\
	\end{array}} \right]\left[ {\begin{array}{*{20}c}
	   {k_1 } & {k_2 }  \\
	\end{array}} \right]} \right\}} \right. \\
	0 & = & \det \left[ {\begin{array}{*{20}c}
	   {s + 4 + k_1 } & {k_2 }  \\
	   { - k_1 } & {s + 11 - k_2 }  \\
	\end{array}} \right] \\
	0 & = & (s + 4 + k_1 )(s + 11 - k_2 ) - (k_2 )( - k_1 ) \\
	0 & = & (s+4+k_1)(s+11-k_2)+k_1k_2
\end{eqnarray*}
\begin{equation}
	\label{eq:3}
	s^2+(15+k_1-k_2)s+(44+11k_1-4k_2)=0
\end{equation}

Now the desired CE is:
\[
\alpha_c(s)=(s+10-j10)(s+10+j10) = 0
\]
\begin{equation}\label{eq:4}
	s^2+20s+200=0
\end{equation}

Therefore matching coefficients in Eqs. (\ref{eq:3}) and (\ref{eq:4}):
\[
\begin{array}{c}
 s^2 :1 = 1 \to {\rm{OK}} \\ 
 s^1 :15 + k_1  - k_2  = 20 \to k_1  - k_2  = 5 \\ 
 s^0 :44 + 11k_1  - 4k_2  = 200 \to 11k_1  - 4k_2  = 156 \\ 
 \end{array}
\]

 

Solving for the $k$'s:
\[
\left[ {\begin{array}{*{20}c}
   1 & { - 1}  \\
   {11} & { - 4}  \\
\end{array}} \right]\left[ {\begin{array}{*{20}c}
   {k_1 }  \\
   {k_2 }  \\
\end{array}} \right] = \left[ {\begin{array}{*{20}c}
   5  \\
   {156}  \\
\end{array}} \right]
\]
% MathType!MTEF!2!1!+-
% faaagaart1ev2aaaKnaaaaWenf2ys9wBH5garuavP1wzZbqedmvETj
% 2BSbqefm0B1jxALjharqqtubsr4rNCHbGeaGqiVu0Je9sqqrpepC0x
% bbL8FesqqrFfpeea0xe9Lq-Jc9vqaqpepm0xbba9pwe9Q8fs0-yqaq
% pepae9pg0FirpepeKkFr0xfr-xfr-xb9Gqpi0dc9adbaqaaeGaciGa
% aiaabeqaamaabaabaaGcbaWaamWaaeaafaWabeGabaaabaGaam4Aam
% aaBaaaleaacaaIXaaabeaaaOqaaiaadUgadaWgaaWcbaGaaGOmaaqa
% baaaaaGccaGLBbGaayzxaaGaeyypa0ZaaSaaaeaacaaIXaaabaGaey
% OeI0IaaGinaiabgUcaRiaaigdacaaIXaaaamaadmaabaqbamqabiGa
% aaqaaiabgkHiTiaaisdaaeaacaaIXaaabaGaeyOeI0IaaGymaiaaig
% daaeaacaaIXaaaaaGaay5waiaaw2faamaadmaabaqbamqabiqaaaqa
% aiaaiwdaaeaacaaIXaGaaGynaiaaiAdaaaaacaGLBbGaayzxaaGaey
% ypa0ZaaSaaaeaacaaIXaaabaGaaG4naaaadaWadaqaauaadeqaceaa
% aeaacaaIXaGaaG4maiaaiAdaaeaacaaIXaGaaGimaiaaigdaaaaaca
% GLBbGaayzxaaGaeyypa0ZaamWaaeaafaWabeGabaaabaGaaGymaiaa
% iMdacaGGUaGaaGinaiaaiMdacaaIYaGaaGyoaaqaaiaaigdacaaI0a
% GaaiOlaiaaisdacaaIYaGaaGyoaaaaaiaawUfacaGLDbaaaaa!5BC6!
\[
\left[ {\begin{array}{*{20}c}
   {k_1 }  \\
   {k_2 }  \\
\end{array}} \right] = \frac{1}{{ - 4 + 11}}\left[ {\begin{array}{*{20}c}
   { - 4} & 1  \\
   { - 11} & 1  \\
\end{array}} \right]\left[ {\begin{array}{*{20}c}
   5  \\
   {156}  \\
\end{array}} \right] = \frac{1}{7}\left[ {\begin{array}{*{20}c}
   {136}  \\
   {101}  \\
\end{array}} \right] = \left[ {\begin{array}{*{20}c}
   {19.429}  \\
   {14.429}  \\
\end{array}} \right]
\]

 

Therefore the required feedback control law is:
% MathType!MTEF!2!1!+-
% faaagaart1ev2aaaKnaaaaWenf2ys9wBH5garuavP1wzZbqedmvETj
% 2BSbqefm0B1jxALjharqqtubsr4rNCHbGeaGqiVu0Je9sqqrpepC0x
% bbL8FesqqrFfpeea0xe9Lq-Jc9vqaqpepm0xbba9pwe9Q8fs0-yqaq
% pepae9pg0FirpepeKkFr0xfr-xfr-xb9Gqpi0dc9adbaqaaeGaciGa
% aiaabeqaamaabaabaaGcbaGaamyDaiabg2da9iaadkhacqGHsislda
% WadaqaauaadeqabiaaaeaacaaIXaGaaGyoaiaac6cacaaI0aGaaGOm
% aiaaiMdaaeaacaaIXaGaaGinaiaac6cacaaI0aGaaGOmaiaaiMdaaa
% aacaGLBbGaayzxaaGaaCiEaaaa!3DCD!
\[
u = r - \left[ {\begin{array}{*{20}c}
   {19.429} & {14.429}  \\
\end{array}} \right]{\bf{x}}
\]



\textbf{COMMENT}
This matching of coefficients can always be done, though it is tedious for $n>3$, \textbf{EXCEPT} in the case of the \emph{Control Canonical Form}.

% subsection example_1 (end)
 
% section finding_the_control_law (end)

\section*{State Feedback in the Case of the Control Canonical Form} % (fold)
\label{sec:state_feedback_in_the_case_of_the_control_canonical_form}


In the control canonical form we have matrices:
% MathType!MTEF!2!1!+-
% faaagaart1ev2aaaKnaaaaWenf2ys9wBH5garuavP1wzZbqedmvETj
% 2BSbqefm0B1jxALjharqqtubsr4rNCHbGeaGqiVu0Je9sqqrpepC0x
% bbL8FesqqrFfpeea0xe9Lq-Jc9vqaqpepm0xbba9pwe9Q8fs0-yqaq
% pepae9pg0FirpepeKkFr0xfr-xfr-xb9Gqpi0dc9adbaqaaeGaciGa
% aiaabeqaamaabaabaaGcbaGaaCyqaiabg2da9maadmaabaqbamqabq
% abaaaaaeaacqGHsislcaWGHbWaaSbaaSqaaiaaigdaaeqaaaGcbaGa
% eyOeI0IaamyyamaaBaaaleaacaaIYaaabeaaaOqaaiabl+Uimbqaai
% abgkHiTiaadggadaWgaaWcbaGaamOBaaqabaaakeaacaaIXaaabaGa
% aGimaaqaaiabl+UimbqaaiaaicdaaeaacaaIWaaabaGaeSy8I8eaba
% GaeS47IWeabaGaeSO7I0eabaGaaGimaaqaaiaaicdaaeaacqWIVlct
% aeaacaaIWaaaaaGaay5waiaaw2faaiaacUdacaaMf8UaaCOqaiabg2
% da9maadmaabaqbamqabqqaaaaabaGaaGymaaqaaiaaicdaaeaacqWI
% UlstaeaacaaIWaaaaaGaay5waiaaw2faaaaa!556B!
\[
{\bf{A}} = \left[ {\begin{array}{*{20}c}
   { - a_1 } & { - a_2 } &  \cdots  & { - a_n }  \\
   1 & 0 &  \cdots  & 0  \\
   0 &  \ddots  &  \cdots  &  \vdots   \\
   0 & 0 &  \cdots  & 0  \\
\end{array}} \right];\quad {\bf{B}} = \left[ {\begin{array}{*{20}c}
   1  \\
   0  \\
    \vdots   \\
   0  \\
\end{array}} \right]
\]
with open loop CE:
\[
\det(s\mathbf{I}-\mathbf{A})=s^n+a_1s^{n-1}+\cdots+a_n=0.
\]

\endinput

%%% Local Variables: 
%%% mode: latex
%%% TeX-master: "notes"
%%% End:
\ifslidesonly
\begin{slide}
	\heading{Control Canonical Form Simplifies Calculation (1)}
   In the control canonical form we have matrices:
% MathType!MTEF!2!1!+-
% faaagaart1ev2aaaKnaaaaWenf2ys9wBH5garuavP1wzZbqedmvETj
% 2BSbqefm0B1jxALjharqqtubsr4rNCHbGeaGqiVu0Je9sqqrpepC0x
% bbL8FesqqrFfpeea0xe9Lq-Jc9vqaqpepm0xbba9pwe9Q8fs0-yqaq
% pepae9pg0FirpepeKkFr0xfr-xfr-xb9Gqpi0dc9adbaqaaeGaciGa
% aiaabeqaamaabaabaaGcbaGaaCyqaiabg2da9maadmaabaqbamqabq
% abaaaaaeaacqGHsislcaWGHbWaaSbaaSqaaiaaigdaaeqaaaGcbaGa
% eyOeI0IaamyyamaaBaaaleaacaaIYaaabeaaaOqaaiabl+Uimbqaai
% abgkHiTiaadggadaWgaaWcbaGaamOBaaqabaaakeaacaaIXaaabaGa
% aGimaaqaaiabl+UimbqaaiaaicdaaeaacaaIWaaabaGaeSy8I8eaba
% GaeS47IWeabaGaeSO7I0eabaGaaGimaaqaaiaaicdaaeaacqWIVlct
% aeaacaaIWaaaaaGaay5waiaaw2faaiaacUdacaaMf8UaaCOqaiabg2
% da9maadmaabaqbamqabqqaaaaabaGaaGymaaqaaiaaicdaaeaacqWI
% UlstaeaacaaIWaaaaaGaay5waiaaw2faaaaa!556B!
\[
{\bf{A}} = \left[ {\begin{array}{*{20}c}
   { - a_1 } & { - a_2 } &  \cdots  & { - a_n }  \\
   1 & 0 &  \cdots  & 0  \\
   0 &  \ddots  &  \cdots  &  \vdots   \\
   0 & 0 &  \cdots  & 0  \\
\end{array}} \right];\quad {\bf{B}} = \left[ {\begin{array}{*{20}c}
   1  \\
   0  \\
    \vdots   \\
   0  \\
\end{array}} \right]
\]
with open loop CE:
\[
\det(s\mathbf{I}-\mathbf{A})=s^n+a_1s^{n-1}+\cdots+a_n=0.
\]

\endinput

%%% Local Variables: 
%%% mode: latex
%%% TeX-master: "notes"
%%% End:
\end{slide}
\fi


The eigenvalues or poles of the combined system are the roots of the CE:
\begin{eqnarray*}
	\det \left[ {s{\bf{I}}_{2n}  - \left[ {\begin{array}{*{20}c}
	   {\left( {{\bf{A}} - {\bf{BK}}} \right)} & {{\bf{BK}}}  \\
	   {\bf{0}} & {\left( {{\bf{A}} - {\bf{LC}}} \right)}  \\
	\end{array}} \right]} \right] & = &  0 \\
	\det \left[ {\begin{array}{*{20}c}
	   {\left[ {s{\bf{I}} - \left( {{\bf{A}} - {\bf{BK}}} \right)} \right]} & { - {\bf{BK}}}  \\
	   {\bf{0}} & {\left[ {s{\bf{I}} - \left( {{\bf{A}} - {\bf{LC}}} \right)} \right]}  \\
	\end{array}} \right] & = & 0
\end{eqnarray*}

\endinput

%%% Local Variables: 
%%% mode: latex
%%% TeX-master: "notes"
%%% End:
\[
{\bf{A}} - {\bf{BK}}  =  \left[ {\begin{array}{*{20}c}
	   {( - a_1  - k_1 )} & {( - a_2  - k_2 )} &  \cdots  & {( - a_n  - k_n )}  \\
	   1 & 0 &  \cdots  &  \vdots   \\
	    \vdots  &  \ddots  &  \cdots  &  \vdots   \\
	   0 & 0 &  \cdots  & 0  \\
	\end{array}} \right]
\]
therefore
\begin{equation}
	\label{eq:5}
	\det(s\mathbf{I}-\mathbf{A}+\mathbf{BK}) = s^n + ( - a_1  - k_1 )s^{n-1} + ( - a_2  - k_2 )s^{n-2} + \cdots +  ( - a_n  - k_n ) = 0.
\end{equation}

\endinput

%%% Local Variables: 
%%% mode: latex
%%% TeX-master: "notes"
%%% End:
\ifslidesonly
\begin{slide}
	\heading{Control Canonical Form (2)}
   The eigenvalues or poles of the combined system are the roots of the CE:
\begin{eqnarray*}
	\det \left[ {s{\bf{I}}_{2n}  - \left[ {\begin{array}{*{20}c}
	   {\left( {{\bf{A}} - {\bf{BK}}} \right)} & {{\bf{BK}}}  \\
	   {\bf{0}} & {\left( {{\bf{A}} - {\bf{LC}}} \right)}  \\
	\end{array}} \right]} \right] & = &  0 \\
	\det \left[ {\begin{array}{*{20}c}
	   {\left[ {s{\bf{I}} - \left( {{\bf{A}} - {\bf{BK}}} \right)} \right]} & { - {\bf{BK}}}  \\
	   {\bf{0}} & {\left[ {s{\bf{I}} - \left( {{\bf{A}} - {\bf{LC}}} \right)} \right]}  \\
	\end{array}} \right] & = & 0
\end{eqnarray*}

\endinput

%%% Local Variables: 
%%% mode: latex
%%% TeX-master: "notes"
%%% End:
\end{slide}
\begin{slide}
	\heading{Control Canonical Form (3)}
   \[
{\bf{A}} - {\bf{BK}}  =  \left[ {\begin{array}{*{20}c}
	   {( - a_1  - k_1 )} & {( - a_2  - k_2 )} &  \cdots  & {( - a_n  - k_n )}  \\
	   1 & 0 &  \cdots  &  \vdots   \\
	    \vdots  &  \ddots  &  \cdots  &  \vdots   \\
	   0 & 0 &  \cdots  & 0  \\
	\end{array}} \right]
\]
therefore
\begin{equation}
	\label{eq:5}
	\det(s\mathbf{I}-\mathbf{A}+\mathbf{BK}) = s^n + ( - a_1  - k_1 )s^{n-1} + ( - a_2  - k_2 )s^{n-2} + \cdots +  ( - a_n  - k_n ) = 0.
\end{equation}

\endinput

%%% Local Variables: 
%%% mode: latex
%%% TeX-master: "notes"
%%% End:
\end{slide}
\fi


 
\begin{itemize}
	\item Rule of thumb: observer poles can be faster than the controller poles (i.e. further from the origin) by a factor of 2 to 6. This makes the effect of the observer dynamics short-term and the overall response is dominated by the controller poles.
	\item If noise/disturbance is present this has an effect on the choice:
	\begin{description}
		\item[Process noise $w$:] $d\mathbf{x}/dt=\mathbf{Ax}+\mathbf{B}u+\mathbf{B}_1 w$
		\item[Sensor noise $v$:]  $y = \mathbf{C}x+v$
		\item[Observer:] $d\hat{\mathbf{x}}=\mathbf{A}\hat{\mathbf{x}}+\mathbf{B}u+\mathbf{L}(y-\mathbf{C}\hat{\mathbf{x}})$
		\item[Error $\mathbf{e}=\mathbf{x}-\hat{\mathbf{x}}$:] $d\mathbf{e}/dt=(\mathbf{A}-\mathbf{LC})\mathbf{e}+\mathbf{B}_1 w - \mathbf{L}v.$
	\end{description}
\end{itemize}

\endinput

%%% Local Variables: 
%%% mode: latex
%%% TeX-master: "notes"
%%% End:
\ifslidesonly
\begin{slide}
	\heading{Control Canonical Form (4)}
   \begin{itemize}
	\item Rule of thumb: observer poles can be faster than the controller poles (i.e. further from the origin) by a factor of 2 to 6. This makes the effect of the observer dynamics short-term and the overall response is dominated by the controller poles.
	\item If noise/disturbance is present this has an effect on the choice:
	\begin{description}
		\item[Process noise $w$:] $d\mathbf{x}/dt=\mathbf{Ax}+\mathbf{B}u+\mathbf{B}_1 w$
		\item[Sensor noise $v$:]  $y = \mathbf{C}x+v$
		\item[Observer:] $d\hat{\mathbf{x}}=\mathbf{A}\hat{\mathbf{x}}+\mathbf{B}u+\mathbf{L}(y-\mathbf{C}\hat{\mathbf{x}})$
		\item[Error $\mathbf{e}=\mathbf{x}-\hat{\mathbf{x}}$:] $d\mathbf{e}/dt=(\mathbf{A}-\mathbf{LC})\mathbf{e}+\mathbf{B}_1 w - \mathbf{L}v.$
	\end{description}
\end{itemize}

\endinput

%%% Local Variables: 
%%% mode: latex
%%% TeX-master: "notes"
%%% End:
\end{slide}
\fi

 
\subsection*{Example 2} % (fold)
\label{sub:example_2}

\textbf{Problem}: Given the system TF:
\[
G(s) = \frac{7}{(s+4)(s+11)}
\]
find the control law for the control canonical form which places the closed loop poles at $s=−10\pm j10$.

\textbf{SOLUTION}:
\[
G(s) = \frac{7}{(s+4)(s+11)} = \frac{7}{(s^2+15s+44)}
\]
 
The control canonical form has matrices:
\[
{\bf{A}} = \left[ {\begin{array}{*{20}c}
   { - 15} & { - 44}  \\
   1 & 0  \\
\end{array}} \right];\quad {\bf{B}} = \left[ {\begin{array}{*{20}c}
   1  \\
   0  \\
\end{array}} \right];\quad {\bf{C}} = \left[ {\begin{array}{*{20}c}
   0 & 7  \\
\end{array}} \right];\quad {\bf{D}} = 0
\]
 
\textbf{NB}:  $\mathbf{C}$ is obtained from the TF numerator $(0s+7)$.
so:    
\[
{\bf{A}} - {\bf{BK}} = \left[ {\begin{array}{*{20}c}
   { - 15 - k_1 } & { - 44 - k_2 }  \\
   1 & 0  \\
\end{array}} \right]
\]
and the closed loop CE is:
\begin{equation}
	s^2+(15+k_1)s+(44+k_2)=0 \label{eq:7}
\end{equation}
The desired CE is:
\[
\alpha_c(s)=(s+10-j10)(s+10+j10) = 0
\]
\begin{equation}\label{eq:8}
	s^2+20s+200=0
\end{equation}
 
Comparing Eqs. (\ref{eq:7})  and  (\ref{eq:8}) gives:
% MathType!MTEF!2!1!+-
% faaagaart1ev2aaaKnaaaaWenf2ys9wBH5garuavP1wzZbqedmvETj
% 2BSbqefm0B1jxALjharqqtubsr4rNCHbGeaGqiVu0Je9sqqrpepC0x
% bbL8FesqqrFfpeea0xe9Lq-Jc9vqaqpepm0xbba9pwe9Q8fs0-yqaq
% pepae9pg0FirpepeKkFr0xfr-xfr-xb9Gqpi0dc9adbaqaaeGaciGa
% aiaabeqaamaabaabaaGcbaGaaGymaiaaiwdacqGHRaWkcaWGRbWaaS
% baaSqaaiaaigdaaeqaaOGaeyypa0JaaGOmaiaaicdacqGHsgIRcaWG
% RbWaaSbaaSqaaiaaigdaaeqaaOGaeyypa0JaaGynaaaa!3A5E!
\[
15 + k_1  = 20 \to k_1  = 5
\]
and
% MathType!MTEF!2!1!+-
% faaagaart1ev2aaaKnaaaaWenf2ys9wBH5garuavP1wzZbqedmvETj
% 2BSbqefm0B1jxALjharqqtubsr4rNCHbGeaGqiVu0Je9sqqrpepC0x
% bbL8FesqqrFfpeea0xe9Lq-Jc9vqaqpepm0xbba9pwe9Q8fs0-yqaq
% pepae9pg0FirpepeKkFr0xfr-xfr-xb9Gqpi0dc9adbaqaaeGaciGa
% aiaabeqaamaabaabaaGcbaGaaGymaiaaiwdacqGHRaWkcaWGRbWaaS
% baaSqaaiaaigdaaeqaaOGaeyypa0JaaGOmaiaaicdacqGHsgIRcaWG
% RbWaaSbaaSqaaiaaigdaaeqaaOGaeyypa0JaaGynaaaa!3A5E!
\[
44 + k_2  = 200 \to k_2  = 156
\]
giving the control law as:   
\[
u = r - \left[ {\begin{array}{*{20}c}
   {5} & {156}  \\
\end{array}} \right]{\bf{x}}
\]

% subsection example_2 (end)Given the system TF:


% section state_feedback_in_the_case_of_the_control_canonical_form (end)

\section*{A Transfer Function Model of State Feedback} % (fold)
\label{sec:a_transfer_function_model_of_state_feedback}

The last example had a system TF with no zeros. In this case it is easy to construct the equivalent classical controller. We had the feedback law:
\[
u=r-5x_1-156x_2
\]
so, taking Laplace transforms:
\[
U(s) = R(s) - 5X_1(s) - 156X_2(s)
\]

Now $y=7x_2$ and $\dot{x}_2=x_1$ therefore $X_2(s)=Y(s)/7$ and $X_1(s)=sX_2(s)=sY(s)/7$. Therefore
\[
	U(s) =  R(s) - \frac{1}{7}(5s+156)Y(s)
\]

\endinput

%%% Local Variables: 
%%% mode: latex
%%% TeX-master: "notes"
%%% End:
\ifslidesonly
\begin{slide}
	\heading{Transfer Function Model of State Feedback (1)}
   The last example had a system TF with no zeros. In this case it is easy to construct the equivalent classical controller. We had the feedback law:
\[
u=r-5x_1-156x_2
\]
so, taking Laplace transforms:
\[
U(s) = R(s) - 5X_1(s) - 156X_2(s)
\]

Now $y=7x_2$ and $\dot{x}_2=x_1$ therefore $X_2(s)=Y(s)/7$ and $X_1(s)=sX_2(s)=sY(s)/7$. Therefore
\[
	U(s) =  R(s) - \frac{1}{7}(5s+156)Y(s)
\]

\endinput

%%% Local Variables: 
%%% mode: latex
%%% TeX-master: "notes"
%%% End:
\end{slide}
\fi

\begin{center}
	\resizebox{300pt}{!}{\includegraphics{pictures/tfmodel.pdf}}
\end{center}

\textbf{Note}: the DC gain is affected -- this could be compensated for by introducing a gain term in series with input $R$.

\endinput

%%% Local Variables: 
%%% mode: latex
%%% TeX-master: "notes"
%%% End:
\ifslidesonly
\begin{slide}
	\heading{Transfer Function Model of State Feedback (2)}
   \begin{center}
	\resizebox{300pt}{!}{\includegraphics{pictures/tfmodel.pdf}}
\end{center}

\textbf{Note}: the DC gain is affected -- this could be compensated for by introducing a gain term in series with input $R$.

\endinput

%%% Local Variables: 
%%% mode: latex
%%% TeX-master: "notes"
%%% End:
\end{slide}
\fi



% section a_transfer_function_model_of_state_feedback (end)

\section*{Ackermann's Formula} % (fold)
\label{sec:ackermann_s_formula}

\subsection*{State Feedback Design for any Form of State Space Model} % (fold)
As stated previously, the derivation of the feedback law is tedious for systems of order  $n>3$  except in the case of the controller canonical form. One approach to the problem is to transform the given model to controller canonical form, derive the control law in terms of these states and then transform back to the original system. Ackermann derived the following formula by this method.
\ifslidesonly
\begin{slide}
   \heading{State Feedback Design for any Form of State Space Model}
\begin{itemize}
	\item As stated previously, the derivation of the feedback law is tedious for systems of order  $n>3$  except in the case of the controller canonical form.
	\item One approach to the problem is to transform the given model to controller canonical form, derive the control law in terms of these states and then transform back to the original system.
	\item Ackermann derived the following formula by this method.
\end{itemize}
\end{slide}
\fi

\subsection*{The formula} % (fold)
\label{sub:the_formula}

Now
\[
U(s) =  - {\bf{K\hat X}}(s) 
\]
so
\[
U (s) =   - {\bf{K}}\left( {{\bf{M}}^{ - 1} {\bf{B}}U(s) + {\bf{M}}^{ - 1} {\bf{L}}Y(s)} \right) 
\]
\[
 \left( {1 + {\bf{KM}}^{ - 1} {\bf{B}}} \right)U(s) =  - {\bf{KM}}^{ - 1} {\bf{L}}Y(s)
\]

\[ 
H\left( s \right) =  - \frac{U(s)}{Y(sßß)} = \frac{{{\bf{KM}}^{ - 1} {\bf{L}}}}{{1 + {\bf{KM}}^{ - 1} {\bf{B}}}} 
\]




\endinput

%%% Local Variables: 
%%% mode: latex
%%% TeX-master: "notes"
%%% End:
\ifslidesonly
\begin{slide}
	\heading{Ackermann's Formula}
   Now
\[
U(s) =  - {\bf{K\hat X}}(s) 
\]
so
\[
U (s) =   - {\bf{K}}\left( {{\bf{M}}^{ - 1} {\bf{B}}U(s) + {\bf{M}}^{ - 1} {\bf{L}}Y(s)} \right) 
\]
\[
 \left( {1 + {\bf{KM}}^{ - 1} {\bf{B}}} \right)U(s) =  - {\bf{KM}}^{ - 1} {\bf{L}}Y(s)
\]

\[ 
H\left( s \right) =  - \frac{U(s)}{Y(sßß)} = \frac{{{\bf{KM}}^{ - 1} {\bf{L}}}}{{1 + {\bf{KM}}^{ - 1} {\bf{B}}}} 
\]




\endinput

%%% Local Variables: 
%%% mode: latex
%%% TeX-master: "notes"
%%% End:
\end{slide}
\fi

The full system response for the state-space model is simply the
sum of the zero-state and zero-input responses:
\begin{eqnarray*}\mathbf{Y}_{\mathrm{full}}(s) &=& \mathbf{Y}_{\mathrm{zs}}(s) +
\mathbf{Y}_{\mathrm{zi}}(s)\\ &=&
\mathbf{C}\Phis{}\left[\mathbf{x}(0)+\mathbf{B}\mathbf{U}(s)\right] + \mathbf{DU}(s).\end{eqnarray*}
\endinput

\ifslidesonly
\begin{slide}
	\heading{Explanation of the Terms}
   The full system response for the state-space model is simply the
sum of the zero-state and zero-input responses:
\begin{eqnarray*}\mathbf{Y}_{\mathrm{full}}(s) &=& \mathbf{Y}_{\mathrm{zs}}(s) +
\mathbf{Y}_{\mathrm{zi}}(s)\\ &=&
\mathbf{C}\Phis{}\left[\mathbf{x}(0)+\mathbf{B}\mathbf{U}(s)\right] + \mathbf{DU}(s).\end{eqnarray*}
\endinput

\end{slide}
\fi

\begin{itemize}
	\item Ackermann's formula is useful for SISO systems of order  $n\le 10$.
	\item $\mathcal{C}$ becomes numerically inaccurate for large $n$.
	\item The system must be \emph{controllable} for $\mathcal{C}^{-1}$ to exist.
\end{itemize}

\endinput

%%% Local Variables: 
%%% mode: latex
%%% TeX-master: "notes"
%%% End:
\ifslidesonly
\begin{slide}
	\heading{Caveats}
   \begin{itemize}
	\item Ackermann's formula is useful for SISO systems of order  $n\le 10$.
	\item $\mathcal{C}$ becomes numerically inaccurate for large $n$.
	\item The system must be \emph{controllable} for $\mathcal{C}^{-1}$ to exist.
\end{itemize}

\endinput

%%% Local Variables: 
%%% mode: latex
%%% TeX-master: "notes"
%%% End:
\end{slide}
\fi

\subsection*{Matlab Function}
From Matlab CST, \verb|help acker|:
\verb|K = ACKER(A,B,P)|  calculates the feedback gain matrix $\mathbf{K}$ such that
the single input system
\[
\dot{\mathbf{x}}=\mathbf{Ax}+\mathbf{B}u
\]
with a feedback law of  $u = -\mathbf{Kx}$  has closed loop poles at the 
values specified in vector $\mathbf{P}$, i.e.,  \texttt{P = eig(A-B*K)}.

\textbf{Note}: This algorithm uses Ackermann's formula.  This method
is NOT numerically reliable and starts to break down rapidly
for problems of order greater than 10, or for weakly controllable
systems.  A warning message is printed if the nonzero closed-loop
poles are greater than 10\% from the desired locations specified 
in $\mathbf{P}$.


\endinput

%%% Local Variables: 
%%% mode: latex
%%% TeX-master: "notes"
%%% End:
\ifslidesonly
\begin{slide}
	\heading{Matlab Function}
   From Matlab CST, \verb|help acker|:
\verb|K = ACKER(A,B,P)|  calculates the feedback gain matrix $\mathbf{K}$ such that
the single input system
\[
\dot{\mathbf{x}}=\mathbf{Ax}+\mathbf{B}u
\]
with a feedback law of  $u = -\mathbf{Kx}$  has closed loop poles at the 
values specified in vector $\mathbf{P}$, i.e.,  \texttt{P = eig(A-B*K)}.

\textbf{Note}: This algorithm uses Ackermann's formula.  This method
is NOT numerically reliable and starts to break down rapidly
for problems of order greater than 10, or for weakly controllable
systems.  A warning message is printed if the nonzero closed-loop
poles are greater than 10\% from the desired locations specified 
in $\mathbf{P}$.


\endinput

%%% Local Variables: 
%%% mode: latex
%%% TeX-master: "notes"
%%% End:
\end{slide}
\fi

% subsection the_formula (end)
 
\subsection*{Example 3} % (fold)
\label{ssec:example_2}

\textbf{Problem}: 
Given:
\[
{\bf{A}} = \left[ {\begin{array}{*{20}c}
   1 & 2  \\
   { - 1} & 1  \\
\end{array}} \right]\quad {\rm{and}}\quad {\bf{B}} = \left[ {\begin{array}{*{20}c}
   1  \\
   { - 2}  \\
\end{array}} \right]
\]
find the feedback vector $\mathbf{K}$ to place the closed loop poles at $s = -1,\ -1$ using Ackermann's formula.

\textbf{SOLUTION}:
\[
\alpha_c(s) = (s + 1)(s + 1) = s^2 + 2s + 1
\]
therefore
\[
\alpha_c(s) = \mathbf{A}s^2 + 2\mathbf{A}s + \mathbf{I}
\]
\[
\alpha _c ({\bf{A}}) = \left[ {\begin{array}{*{20}c}
   { - 1} & 4  \\
   { - 2} & { - 1}  \\
\end{array}} \right] + 2\left[ {\begin{array}{*{20}c}
   1 & 2  \\
   { - 1} & 1  \\
\end{array}} \right] + \left[ {\begin{array}{*{20}c}
   1 & 0  \\
   0 & 1  \\
\end{array}} \right] = \left[ {\begin{array}{*{20}c}
   2 & 8  \\
   { - 4} & 2  \\
\end{array}} \right]
\]

\[
{\bf{AB}} = \left[ {\begin{array}{*{20}c}
   { - 3}  \\
   { - 3}  \\
\end{array}} \right];\quad \mathcal{C} = \left[ {{\bf{A}} \vdots {\bf{AB}}} \right] = \left[ {\begin{array}{*{20}c}
   1 & { - 3}  \\
   { - 2} & { - 3}  \\
\end{array}} \right]
\]

\begin{eqnarray*}
	{\bf{K}} & = & \left[ {\begin{array}{*{20}c}
	   0 &  \cdots  & 0 & 1  \\
	\end{array}} \right]\mathcal{C}^{ - 1} \alpha _c ({\bf{A}}) \\
	& = & \left[ {\begin{array}{*{20}c}
	   0 & 1  \\
	\end{array}} \right]\left[ {\begin{array}{*{20}c}
	   1 & { - 3}  \\
	   { - 2} & { - 3}  \\
	\end{array}} \right]^{ - 1} \left[ {\begin{array}{*{20}c}
	   2 & 8  \\
	   { - 4} & 2  \\
	\end{array}} \right] \\
	& = & \left[ {\begin{array}{*{20}c}
	   0 & 1  \\
	\end{array}} \right]\frac{\left[ {\begin{array}{*{20}c}
	   { -3 } & { 3 }  \\
	   { 2 } & { 1 }  \\
	\end{array}} \right]}{-3-(+6)} \left[ {\begin{array}{*{20}c}
	   2 & 8  \\
	   { - 4} & 2  \\
	\end{array}} \right] \\
	& = & \left[ {\begin{array}{*{20}c}
	   0 & 1  \\
	\end{array}} \right]\frac{1}{-9}\left[ \begin{array}{*{20}c}
	   { -18 } & { -18 }  \\
	   { 0 } & { 18 }  \\
	\end{array} \right] \\
	& = & \left[ {\begin{array}{*{20}c}
	   0 & 1  \\
	\end{array}} \right]\left[ \begin{array}{*{20}c}
	   { 2 } & { 2 }  \\
	   { 0 } & { -2 }  \\
	\end{array} \right] \\
	& = & \left[ {\begin{array}{*{20}c}
	   0 & -2  \\
	\end{array}} \right]
\end{eqnarray*}
% subsection example_3 (end)

\subsection*{Solution in Matlab} % (fold)
\label{sub:solution_in_matlab}
\begin{verbatim}
% Using the formula
A = [1 2; -1 1]; B = [1; -2];
alpha_c = A * A + 2 * A + eye(2);
K = [0 1] * inv(ctrb(A, B)) * alpha_c

% Using the function Acker
P = [-1, -1] % vector of desired pole locations
Ka = acker(A, B, P)
\end{verbatim}
\ifslidesonly
\begin{slide}
   \heading{Solution in Matlab(1)}
\begin{verbatim}
A = [1 2; -1 1]; B = [1; -2];
alpha_c = A * A + 2 * A + eye(2);
K = [0 1] * inv(ctrb(A, B)) * alpha_c
\end{verbatim}
\end{slide}
\begin{slide}
   \heading{Solution in Matlab(2)}
\begin{verbatim}
% Using the function Acker
P = [-1, -1] % vector of desired pole locations
Ka = acker(A, B, P)
\end{verbatim}
\end{slide}
\fi 
% subsection solution_in_matlab (end)


 


 

% section ackermann_s_formula (end)

\section*{Effect of State Feedback on the Closed Loop Zeros} % (fold)
\label{sec:effect_of_state_feedback_on_the_closed_loop_zeros}

If $\mathbf{M}$ is square and $\mathbf{V}$ is a row vector and $\mathbf{W}$ is a column vector then,
\[
\mathbf{VM}^{-1}\mathbf{W}=\frac{\det\left(\mathbf{M}+\mathbf{WV}\right)}{\det{\mathbf{M}}}-1
\]
 


\endinput

%%% Local Variables: 
%%% mode: latex
%%% TeX-master: "notes"
%%% End:
\ifslidesonly
\begin{slide}
	\heading{Closed-Loop System}
   If $\mathbf{M}$ is square and $\mathbf{V}$ is a row vector and $\mathbf{W}$ is a column vector then,
\[
\mathbf{VM}^{-1}\mathbf{W}=\frac{\det\left(\mathbf{M}+\mathbf{WV}\right)}{\det{\mathbf{M}}}-1
\]
 


\endinput

%%% Local Variables: 
%%% mode: latex
%%% TeX-master: "notes"
%%% End:
\end{slide}
\fi



By analogy with previous work (see Section `\emph{Some Important Properties}', in Lecture 15), the TF from reference input $r$  to output $y$  is:
\[
\frac{{Y(s)}}{{R(s)}} = \frac{{\det \left[ {\begin{array}{*{20}c}
   {(s{\bf{I}} - {\bf{A}} + {\bf{BK}})} & { - {\bf{B}}}  \\
   {({\bf{C}} - {\bf{DK}})} & {\bf{D}}  \\
\end{array}} \right]}}{{\det (s{\bf{I}} - {\bf{A}} + {\bf{BK}})}}
\]

The closed loop TF zeros are determined by the numerator determinant.


\endinput

%%% Local Variables: 
%%% mode: latex
%%% TeX-master: "notes"
%%% End:
\ifslidesonly
\begin{slide}
	\heading{Closed-loop Transfer Function}
   By analogy with previous work (see Section `\emph{Some Important Properties}', in Lecture 15), the TF from reference input $r$  to output $y$  is:
\[
\frac{{Y(s)}}{{R(s)}} = \frac{{\det \left[ {\begin{array}{*{20}c}
   {(s{\bf{I}} - {\bf{A}} + {\bf{BK}})} & { - {\bf{B}}}  \\
   {({\bf{C}} - {\bf{DK}})} & {\bf{D}}  \\
\end{array}} \right]}}{{\det (s{\bf{I}} - {\bf{A}} + {\bf{BK}})}}
\]

The closed loop TF zeros are determined by the numerator determinant.


\endinput

%%% Local Variables: 
%%% mode: latex
%%% TeX-master: "notes"
%%% End:
\end{slide}
\fi

Adding $\mathbf{K}$ times the $2^\mathrm{nd}$ column to the first cancels terms whilst leaving the determinant unchanged.

The new form for the TF is:
\[
\frac{{Y(s)}}{{R(s)}} = \frac{{\det \left[ {\begin{array}{*{20}c}
   {(s{\bf{I}} - {\bf{A}})} & { - {\bf{B}}}  \\
   {\bf{C}} & {\bf{D}}  \\
\end{array}} \right]}}{{\det (s{\bf{I}} - {\bf{A}} + {\bf{BK}})}}
\]

Notice now that numerator is identical to that of the open loop TF. This implies that the state feedback control has left the open loop zeros unchanged. The different denominator is due to the feedback action which alters the pole positions as required.

\endinput

%%% Local Variables: 
%%% mode: latex
%%% TeX-master: "notes"
%%% End:
\ifslidesonly
\begin{slide}
	\heading{Effect of State-Feedback on Closed-Loop Zeros}
   Adding $\mathbf{K}$ times the $2^\mathrm{nd}$ column to the first cancels terms whilst leaving the determinant unchanged.

The new form for the TF is:
\[
\frac{{Y(s)}}{{R(s)}} = \frac{{\det \left[ {\begin{array}{*{20}c}
   {(s{\bf{I}} - {\bf{A}})} & { - {\bf{B}}}  \\
   {\bf{C}} & {\bf{D}}  \\
\end{array}} \right]}}{{\det (s{\bf{I}} - {\bf{A}} + {\bf{BK}})}}
\]

Notice now that numerator is identical to that of the open loop TF. This implies that the state feedback control has left the open loop zeros unchanged. The different denominator is due to the feedback action which alters the pole positions as required.

\endinput

%%% Local Variables: 
%%% mode: latex
%%% TeX-master: "notes"
%%% End:
\end{slide}
\fi


% section effect_of_state_feedback_on_the_closed_loop_zeros (end)

\section*{Effect of Zero Locations on the Feedback Gains} % (fold)
\label{sec:effect_of_zero_locations_on_the_feedback_gains}


When a zero is close to a pole in the TF there is a marked increase in the feedback gains. This effect is best illustrated with an example.

Given a system with a TF,
\[
\frac{Y(s)}{U(s)}=\frac{s-z}{s-p}=1+\frac{p-z}{s-p}
\]
find the control law to move the pole to $p_c$.



\endinput

%%% Local Variables: 
%%% mode: latex
%%% TeX-master: "notes"
%%% End:
\ifslidesonly
\begin{slide}
	\heading{Effect of Zero Locations on the Feedback Gains -- Example}
   When a zero is close to a pole in the TF there is a marked increase in the feedback gains. This effect is best illustrated with an example.

Given a system with a TF,
\[
\frac{Y(s)}{U(s)}=\frac{s-z}{s-p}=1+\frac{p-z}{s-p}
\]
find the control law to move the pole to $p_c$.



\endinput

%%% Local Variables: 
%%% mode: latex
%%% TeX-master: "notes"
%%% End:
\end{slide}
\fi

Using the observer canonical form,
\[
\dot{x}=px+(p - z)u,\ y = x + u.
\]

Design the feedback to move the closed-loop pole to $p_c$.
Now,
\[
\mathbf{A}=p;\ \mathbf{B}=p-z;\ \mathbf{K}=k_1.
\]
Desired CE polynomial: $\alpha_c(s)=s-p_c$. Actual CE polynomial: 
$\det(s\mathbf{I}-\mathbf{A}-\mathbf{BK}) = s - p + (p - z)k_1.$

Comparing the constant:
\begin{eqnarray*}
	-p_c & = & -p + (p - z)k_1 \\
	k_1 & = & \frac{p-p_c}{p-z}
\end{eqnarray*}


\endinput

%%% Local Variables: 
%%% mode: latex
%%% TeX-master: "notes"
%%% End:
\ifslidesonly
\begin{slide}
	\heading{Solution}
   Using the observer canonical form,
\[
\dot{x}=px+(p - z)u,\ y = x + u.
\]

Design the feedback to move the closed-loop pole to $p_c$.
Now,
\[
\mathbf{A}=p;\ \mathbf{B}=p-z;\ \mathbf{K}=k_1.
\]
Desired CE polynomial: $\alpha_c(s)=s-p_c$. Actual CE polynomial: 
$\det(s\mathbf{I}-\mathbf{A}-\mathbf{BK}) = s - p + (p - z)k_1.$

Comparing the constant:
\begin{eqnarray*}
	-p_c & = & -p + (p - z)k_1 \\
	k_1 & = & \frac{p-p_c}{p-z}
\end{eqnarray*}


\endinput

%%% Local Variables: 
%%% mode: latex
%%% TeX-master: "notes"
%%% End:
\end{slide}
\fi

\ifslidesonly
\begin{slide}
   \heading{Comments on Solution}
\[
	k_1 =  \frac{p-p_c}{p-z}
\]
We know from previous work that the denominator, corresponding to the closed loop CE of the overall system must, from the separation principle, be equivalent to:
\[
\alpha_e(s)\alpha_c(s)
\]
therefore,
\[
\frac{{Y(s)}}{{R(s)}} = \frac{{\alpha _e (s)\alpha _z (s)}}{{\alpha _e (s)\alpha _c (s)}} = \frac{{\alpha _z (s)}}{{\alpha _c (s)}}
\]



\endinput

%%% Local Variables: 
%%% mode: latex
%%% TeX-master: "notes"
%%% End:
\end{slide}
\fi
We know from previous work that the denominator, corresponding to the closed loop CE of the overall system must, from the separation principle, be equivalent to:
\[
\alpha_e(s)\alpha_c(s)
\]
therefore,
\[
\frac{{Y(s)}}{{R(s)}} = \frac{{\alpha _e (s)\alpha _z (s)}}{{\alpha _e (s)\alpha _c (s)}} = \frac{{\alpha _z (s)}}{{\alpha _c (s)}}
\]



\endinput

%%% Local Variables: 
%%% mode: latex
%%% TeX-master: "notes"
%%% End:
	
Large feedback control gains are required if:
\begin{enumerate}
	\item There exist almost cancelling pole-zero pairs in the open loop TF, making the system almost uncontrollable.
\begin{center}
	\resizebox{300pt}{!}{\includegraphics{pictures/zeroloc.pdf}}
\end{center}
	(Notice that in this condition the input $u$ is almost disconnected from the integrator for the state.)
	\item One tries to move the poles a long way, ($|p-p_c|$ large). 
	
	This imposes a practical limit on how arbitrarily 
	the poles can be placed. You cannot make a slow system fast without using large gains requiring powerful, 
	expensive actuators to force the plant response. Indeed, excessively large forces may destroy the plant.
\end{enumerate}

\endinput

%%% Local Variables: 
%%% mode: latex
%%% TeX-master: "notes"
%%% End:
\ifslidesonly
\begin{slide}
	\heading{Effect of Zero Locations on the Feedback Gains}
   Large feedback control gains are required if:
\begin{enumerate}
	\item There exist almost cancelling pole-zero pairs in the open loop TF, making the system almost uncontrollable.
\begin{center}
	\resizebox{300pt}{!}{\includegraphics{pictures/zeroloc.pdf}}
\end{center}
	(Notice that in this condition the input $u$ is almost disconnected from the integrator for the state.)
	\item One tries to move the poles a long way, ($|p-p_c|$ large). 
	
	This imposes a practical limit on how arbitrarily 
	the poles can be placed. You cannot make a slow system fast without using large gains requiring powerful, 
	expensive actuators to force the plant response. Indeed, excessively large forces may destroy the plant.
\end{enumerate}

\endinput

%%% Local Variables: 
%%% mode: latex
%%% TeX-master: "notes"
%%% End:
\end{slide}
\fi


 
% section effect_of_zero_locations_on_the_feedback_gains (end)

\section*{Pole Selection for Good Design} % (fold)
\label{sec:pole_selection_for_good_design}

In this configuration the observer is driven by:
\[
\tilde E =  Y + R
\] 
Hence the observer dynamics are:
\[
s\hat{\mathbf{X}}=\mathbf{A}\hat{\mathbf{X}}+\mathbf{B}U+\mathbf{L}(\tilde E - \mathbf{C}\hat{\mathbf{X}})
\]
therefore,
\begin{eqnarray*}
	\underbrace {\left( {s{\bf{I}} - {\bf{A}} + {\bf{LC}}} \right)}_{\bf{M}}{\bf{\hat X}} & = & {\bf{B}}U + {\bf{L}}\tilde E \\
	{\bf{\hat X}} & = & {\bf{M}}^{ - 1} {\bf{B}}U + {\bf{M}}^{ - 1} {\bf{L}}\tilde E
\end{eqnarray*}

Now $U=-\mathbf{K}\hat{\mathbf{X}}$, therefore
\begin{eqnarray*}
	U & = &  - \bf{KM}^{ - 1} {\bf{B}}U - {\bf{KM}}^{ - 1} {\bf{L}}\tilde E \\
	U & = &  - \frac{{{\bf{KM}}^{ - 1} {\bf{L}}}}{{1 + {\bf{KM}}^{ - 1} {\bf{B}}}}\tilde E \\
	H\left( s \right) & = &  - \frac{U}{{\tilde E}} = \frac{{{\bf{KM}}^{ - 1} {\bf{L}}}}{{1 + {\bf{KM}}^{ - 1} {\bf{B}}}}
\end{eqnarray*}

Here $H(s)$ is the same as before:
\[
\frac{\alpha_2(s)}{\alpha_1(s)}
\]
where $\alpha_1(s)=\det(\mathbf{M}+\mathbf{BK})$ and $\alpha_2(s)=\det(\mathbf{M}+\mathbf{LK})-\det{M}$.
 
The overall TF is:
\begin{eqnarray*}
	\frac{Y(s)}{R(s)} &=& \frac{G(s)}{1+G(s)H(s)}= \frac{\frac{\alpha_z(s)}{\alpha_p(s)}\times\frac{\alpha_2(s)}{\alpha_1(s)}}{1+\frac{\alpha_z(s)}{\alpha_p(s)}\times\frac{\alpha_2(s)}{\alpha_1(s)}} \\
	\frac{Y(s)}{R(s)} &=& \frac{\alpha_z(s)\alpha_2(s)}{\alpha_p(s)\alpha_1(s)+\alpha_z(s)\alpha_2(s)}=\frac{\alpha_z\alpha_2}{\alpha_c\alpha_e}
\end{eqnarray*}
 

In this case we see that the overall TF contains the poles of the observer as well as the controller.
Whereas in the normal position changes in the reference input do not excite the error dynamics of the observer, in this configuration they do.
As a result the difference between the observer and the plant states is affected during operation and take further time to settle down.

\endinput

%%% Local Variables: 
%%% mode: latex
%%% TeX-master: "notes"
%%% End:
\ifslidesonly
\begin{slide}
	\heading{Pole Selection for Good Design (1)}
   In this configuration the observer is driven by:
\[
\tilde E =  Y + R
\] 
Hence the observer dynamics are:
\[
s\hat{\mathbf{X}}=\mathbf{A}\hat{\mathbf{X}}+\mathbf{B}U+\mathbf{L}(\tilde E - \mathbf{C}\hat{\mathbf{X}})
\]
therefore,
\begin{eqnarray*}
	\underbrace {\left( {s{\bf{I}} - {\bf{A}} + {\bf{LC}}} \right)}_{\bf{M}}{\bf{\hat X}} & = & {\bf{B}}U + {\bf{L}}\tilde E \\
	{\bf{\hat X}} & = & {\bf{M}}^{ - 1} {\bf{B}}U + {\bf{M}}^{ - 1} {\bf{L}}\tilde E
\end{eqnarray*}

Now $U=-\mathbf{K}\hat{\mathbf{X}}$, therefore
\begin{eqnarray*}
	U & = &  - \bf{KM}^{ - 1} {\bf{B}}U - {\bf{KM}}^{ - 1} {\bf{L}}\tilde E \\
	U & = &  - \frac{{{\bf{KM}}^{ - 1} {\bf{L}}}}{{1 + {\bf{KM}}^{ - 1} {\bf{B}}}}\tilde E \\
	H\left( s \right) & = &  - \frac{U}{{\tilde E}} = \frac{{{\bf{KM}}^{ - 1} {\bf{L}}}}{{1 + {\bf{KM}}^{ - 1} {\bf{B}}}}
\end{eqnarray*}

Here $H(s)$ is the same as before:
\[
\frac{\alpha_2(s)}{\alpha_1(s)}
\]
where $\alpha_1(s)=\det(\mathbf{M}+\mathbf{BK})$ and $\alpha_2(s)=\det(\mathbf{M}+\mathbf{LK})-\det{M}$.
 
The overall TF is:
\begin{eqnarray*}
	\frac{Y(s)}{R(s)} &=& \frac{G(s)}{1+G(s)H(s)}= \frac{\frac{\alpha_z(s)}{\alpha_p(s)}\times\frac{\alpha_2(s)}{\alpha_1(s)}}{1+\frac{\alpha_z(s)}{\alpha_p(s)}\times\frac{\alpha_2(s)}{\alpha_1(s)}} \\
	\frac{Y(s)}{R(s)} &=& \frac{\alpha_z(s)\alpha_2(s)}{\alpha_p(s)\alpha_1(s)+\alpha_z(s)\alpha_2(s)}=\frac{\alpha_z\alpha_2}{\alpha_c\alpha_e}
\end{eqnarray*}
 

In this case we see that the overall TF contains the poles of the observer as well as the controller.
Whereas in the normal position changes in the reference input do not excite the error dynamics of the observer, in this configuration they do.
As a result the difference between the observer and the plant states is affected during operation and take further time to settle down.

\endinput

%%% Local Variables: 
%%% mode: latex
%%% TeX-master: "notes"
%%% End:
\end{slide}
\fi


In this configuration the observer is driven by:
\[
\tilde E =  Y + R
\] 
Hence the observer dynamics are:
\[
s\hat{\mathbf{X}}=\mathbf{A}\hat{\mathbf{X}}+\mathbf{B}U+\mathbf{L}(\tilde E - \mathbf{C}\hat{\mathbf{X}})
\]
therefore,
\begin{eqnarray*}
	\underbrace {\left( {s{\bf{I}} - {\bf{A}} + {\bf{LC}}} \right)}_{\bf{M}}{\bf{\hat X}} & = & {\bf{B}}U + {\bf{L}}\tilde E \\
	{\bf{\hat X}} & = & {\bf{M}}^{ - 1} {\bf{B}}U + {\bf{M}}^{ - 1} {\bf{L}}\tilde E
\end{eqnarray*}



\endinput

%%% Local Variables: 
%%% mode: latex
%%% TeX-master: "notes"
%%% End:
\ifslidesonly
\begin{slide}
	\heading{Pole Selection for Good Design (2)}
   In this configuration the observer is driven by:
\[
\tilde E =  Y + R
\] 
Hence the observer dynamics are:
\[
s\hat{\mathbf{X}}=\mathbf{A}\hat{\mathbf{X}}+\mathbf{B}U+\mathbf{L}(\tilde E - \mathbf{C}\hat{\mathbf{X}})
\]
therefore,
\begin{eqnarray*}
	\underbrace {\left( {s{\bf{I}} - {\bf{A}} + {\bf{LC}}} \right)}_{\bf{M}}{\bf{\hat X}} & = & {\bf{B}}U + {\bf{L}}\tilde E \\
	{\bf{\hat X}} & = & {\bf{M}}^{ - 1} {\bf{B}}U + {\bf{M}}^{ - 1} {\bf{L}}\tilde E
\end{eqnarray*}



\endinput

%%% Local Variables: 
%%% mode: latex
%%% TeX-master: "notes"
%%% End:
\end{slide}
\fi

\subsubsection*{Comments} % (fold)
\label{ssub:comments}

Now $U=-\mathbf{K}\hat{\mathbf{X}}$, therefore
\begin{eqnarray*}
	U & = &  - \bf{KM}^{ - 1} {\bf{B}}U - {\bf{KM}}^{ - 1} {\bf{L}}\tilde E \\
	U & = &  - \frac{{{\bf{KM}}^{ - 1} {\bf{L}}}}{{1 + {\bf{KM}}^{ - 1} {\bf{B}}}}\tilde E \\
	H\left( s \right) & = &  - \frac{U}{{\tilde E}} = \frac{{{\bf{KM}}^{ - 1} {\bf{L}}}}{{1 + {\bf{KM}}^{ - 1} {\bf{B}}}}
\end{eqnarray*}


\endinput

%%% Local Variables: 
%%% mode: latex
%%% TeX-master: "notes"
%%% End:
\ifslidesonly
\begin{slide}
	\heading{Comments}
   Now $U=-\mathbf{K}\hat{\mathbf{X}}$, therefore
\begin{eqnarray*}
	U & = &  - \bf{KM}^{ - 1} {\bf{B}}U - {\bf{KM}}^{ - 1} {\bf{L}}\tilde E \\
	U & = &  - \frac{{{\bf{KM}}^{ - 1} {\bf{L}}}}{{1 + {\bf{KM}}^{ - 1} {\bf{B}}}}\tilde E \\
	H\left( s \right) & = &  - \frac{U}{{\tilde E}} = \frac{{{\bf{KM}}^{ - 1} {\bf{L}}}}{{1 + {\bf{KM}}^{ - 1} {\bf{B}}}}
\end{eqnarray*}


\endinput

%%% Local Variables: 
%%% mode: latex
%%% TeX-master: "notes"
%%% End:
\end{slide}
\fi


A most effective technique is to use optimal control to achieve a compromise between control effort, $u$, and error, $e$.
i.e. Find the feedback vector $\mathbf{K}$ such as to minimise,
\[
J=\int_0^{\infty}\left(e^2+\frac{u^2}{k}\right) dt
\]
where the choice of the parameter $k$ determines the required compromise between,
\begin{itemize}
	\item High Accuracy for High Control Effort (use a large value for $k$)
	\item Lower Accuracy for Reduced Control Effort (use a smaller value for $k$)
\end{itemize}

\endinput

%%% Local Variables: 
%%% mode: latex
%%% TeX-master: "notes"
%%% End:
\ifslidesonly
\begin{slide}
	\heading{Optimal State Feedback}
   A most effective technique is to use optimal control to achieve a compromise between control effort, $u$, and error, $e$.
i.e. Find the feedback vector $\mathbf{K}$ such as to minimise,
\[
J=\int_0^{\infty}\left(e^2+\frac{u^2}{k}\right) dt
\]
where the choice of the parameter $k$ determines the required compromise between,
\begin{itemize}
	\item High Accuracy for High Control Effort (use a large value for $k$)
	\item Lower Accuracy for Reduced Control Effort (use a smaller value for $k$)
\end{itemize}

\endinput

%%% Local Variables: 
%%% mode: latex
%%% TeX-master: "notes"
%%% End:
\end{slide}
\fi


% subsubsection comments (end)



% section pole_selection_for_good_design (end)


\ifslidesonly
\begin{slide}
   \heading{Summary of this Lecture}
   \begin{itemize}
   	\item Finding the control law
   	\item State feedback for controller canonical form
   	\item Transfer function model
   	\item Ackermann's formula
   	\item Effect of state feedback on closed-loop zeros
   	\item Effect of plant zeros on the feedback gains
   	\item Pole-selection for good design
   \end{itemize}
\end{slide}
\fi


%----------------------------------------------------------------
% The end of notes
% ----------------------------------------------------------------
\endinput

%%% Local Variables: 
%%% mode: latex
%%% TeX-master: t
%%% End: 

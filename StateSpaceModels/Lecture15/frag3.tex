Let us now transform the generalized form of the state equations
obtained in the last lecture.
\begin{eqnarray*}
  \frac{d\mathbf{x}(t)}{dt} &=&
  \mathbf{A}\mathbf{x}(t)+\mathbf{B}\mathbf{u}(t)\\
  \mathbf{y}(t)&=&\mathbf{C}\mathbf{x}(t)+\mathbf{D}\mathbf{u}(t)
\end{eqnarray*}
Applying the Laplace transform to both sides of this matrix
equation gives the transform equations
\begin{eqnarray*}
  s\mathbf{X}(s)-\mathbf{x}(0) &=&
  \mathbf{A}\mathbf{X}(s)+\mathbf{B}\mathbf{U}(s)\\
  \mathbf{Y}(s)&=&\mathbf{C}\mathbf{X}(s)+\mathbf{D}\mathbf{U}(s)
\end{eqnarray*}
where $\mathbf{x}(0)$ is the vector of initial conditions vector
of the states; $\mathbf{X}(s)$ is the state transform vector;
$\mathbf{U}(s)$ input transform vector; $\mathbf{Y}(s)$ is output
transform vector.

\endinput

%%% Local Variables: 
%%% mode: latex
%%% TeX-master: "notes"
%%% End: 

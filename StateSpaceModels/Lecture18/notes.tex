%% State Space Modelling of Dynamic Systems
%% Lecture 18: General Solution of a State Space System
\def\FileDate{10/02/01}
\def\FileVersion{1.0}
% ----------------------------------------------------------------
% Notes pages *********************************************************
% ----------------------------------------------------------------

In this lecture we conclude our introduction to state space systems by developing a method that can be used to solve any linear time invariant (LTI) system using the state space model. This will put on a formal mathematical footing the approach described in Lecture 15. To do this we need to derive the Taylor series for the state matrix and then use the results from the previous two lectures to formally derive the state transmission matrix $\mathbf{\phi}(t)$ from the resolvant matrix $\mathbf{\Phi}(\lambda) = |\lambda\mathbf{I}-\mathbf{A}|$. This will lead is to a general solution which we will illustrate with examples and compare with the solution developed in the last lecture and the solution obtained by inverting the system transfer function. As before, we will refer to Matlab where relevant.

Once we have completed this material, we will be in a position to move on to look at state-space methods for control system design.

\section*{The Matrix Exponential Function}

In mathematics, the Taylor series is a representation of a function as an infinite sum of terms calculated from the values of its derivatives at a single point. It is named after the English mathematician Brook Taylor. It is common practice to use a finite number of terms of the series to approximate a function. The Taylor series may be regarded as the limit of the Taylor polynomials.\footnote{Definition from `Taylor Series', Wikipedia.}

We use Taylor series to approximate general functions, and here we adapt the Taylor series to discover the form of a function of a matrix.
 
Therefore the dynamics of the combined system are:
\[
\left[ {\begin{array}{*{20}c}
   {{\bf{\dot x}}}  \\
   {{\bf{\dot e}}}  \\
\end{array}} \right] = \left[ {\begin{array}{*{20}c}
   {\left( {{\bf{A}} - {\bf{BK}}} \right)} & {{\bf{BK}}}  \\
   {\bf{0}} & {\left( {{\bf{A}} - {\bf{LC}}} \right)}  \\
\end{array}} \right]\left[ {\begin{array}{*{20}c}
   {\bf{x}}  \\
   {\bf{e}}  \\
\end{array}} \right] + \left[ {\begin{array}{*{20}c}
   {\bf{B}}  \\
   {\bf{0}}  \\
	\end{array}} \right]r
\]

\endinput

%%% Local Variables: 
%%% mode: latex
%%% TeX-master: "notes"
%%% End:
\ifslidesonly
\begin{slide}
	\heading{Taylor series}
   Therefore the dynamics of the combined system are:
\[
\left[ {\begin{array}{*{20}c}
   {{\bf{\dot x}}}  \\
   {{\bf{\dot e}}}  \\
\end{array}} \right] = \left[ {\begin{array}{*{20}c}
   {\left( {{\bf{A}} - {\bf{BK}}} \right)} & {{\bf{BK}}}  \\
   {\bf{0}} & {\left( {{\bf{A}} - {\bf{LC}}} \right)}  \\
\end{array}} \right]\left[ {\begin{array}{*{20}c}
   {\bf{x}}  \\
   {\bf{e}}  \\
\end{array}} \right] + \left[ {\begin{array}{*{20}c}
   {\bf{B}}  \\
   {\bf{0}}  \\
	\end{array}} \right]r
\]

\endinput

%%% Local Variables: 
%%% mode: latex
%%% TeX-master: "notes"
%%% End:
\end{slide}
\fi

 


\begin{center}
	\begin{tabular}{|c|c|c|}
	\hline
	\textbf{Sub-system} & \textbf{Controllable?} & \textbf{Observable?}\\
	\hline
	$\mathbf{S}_1$ & Yes & No\\
	\hline
	$\mathbf{S}_2$ & Yes & Yes\\
	\hline
	$\mathbf{S}_3$ & No & Yes\\
	\hline
	$\mathbf{S}_4$ & No & No\\
	\hline
	\end{tabular}	
\end{center}

\endinput

%%% Local Variables: 
%%% mode: latex
%%% TeX-master: "notes"
%%% End:
\ifslidesonly
\begin{slide}
	\heading{Using the Similarity Transform}
   
\begin{center}
	\begin{tabular}{|c|c|c|}
	\hline
	\textbf{Sub-system} & \textbf{Controllable?} & \textbf{Observable?}\\
	\hline
	$\mathbf{S}_1$ & Yes & No\\
	\hline
	$\mathbf{S}_2$ & Yes & Yes\\
	\hline
	$\mathbf{S}_3$ & No & Yes\\
	\hline
	$\mathbf{S}_4$ & No & No\\
	\hline
	\end{tabular}	
\end{center}

\endinput

%%% Local Variables: 
%%% mode: latex
%%% TeX-master: "notes"
%%% End:
\end{slide}
\fi


 
\begin{itemize}
	\item We shall determine the classical compensator TF which is equivalent to the combined observer-controller. 
	\item This is simpler to do if we remove the reference input, for the time being.
\end{itemize}
\begin{center}
	\resizebox{300pt}{!}{\includegraphics{pictures/fig1.pdf}}
\end{center}
\endinput

%%% Local Variables: 
%%% mode: latex
%%% TeX-master: "notes"
%%% End:
\ifslidesonly
\begin{slide}
	\heading{Matrix Function}
   \begin{itemize}
	\item We shall determine the classical compensator TF which is equivalent to the combined observer-controller. 
	\item This is simpler to do if we remove the reference input, for the time being.
\end{itemize}
\begin{center}
	\resizebox{300pt}{!}{\includegraphics{pictures/fig1.pdf}}
\end{center}
\endinput

%%% Local Variables: 
%%% mode: latex
%%% TeX-master: "notes"
%%% End:
\end{slide}
\fi

 
In particular, the solution of:
\[
\frac{d\mathbf{x}}{dt}=\mathbf{Ax}
\]
given $\mathbf{x}=\mathbf{x}_0$ at $t=0$ is:
\[
\mathbf{x}=e^{\mathbf{A}t}\mathbf{x}_0
\]
               
The term  $\mathbf{\phi}(t) = e^{\mathbf{A}t}$  is known as the state transition matrix because it shows how time, $t$, transforms the initial state vector into the present one.

\endinput

%%% Local Variables: 
%%% mode: latex
%%% TeX-master: "notes"
%%% End:
\ifslidesonly
\begin{slide}
	\heading{Transition Matrix}
   In particular, the solution of:
\[
\frac{d\mathbf{x}}{dt}=\mathbf{Ax}
\]
given $\mathbf{x}=\mathbf{x}_0$ at $t=0$ is:
\[
\mathbf{x}=e^{\mathbf{A}t}\mathbf{x}_0
\]
               
The term  $\mathbf{\phi}(t) = e^{\mathbf{A}t}$  is known as the state transition matrix because it shows how time, $t$, transforms the initial state vector into the present one.

\endinput

%%% Local Variables: 
%%% mode: latex
%%% TeX-master: "notes"
%%% End:
\end{slide}
\fi


In the control canonical form we have matrices:
% MathType!MTEF!2!1!+-
% faaagaart1ev2aaaKnaaaaWenf2ys9wBH5garuavP1wzZbqedmvETj
% 2BSbqefm0B1jxALjharqqtubsr4rNCHbGeaGqiVu0Je9sqqrpepC0x
% bbL8FesqqrFfpeea0xe9Lq-Jc9vqaqpepm0xbba9pwe9Q8fs0-yqaq
% pepae9pg0FirpepeKkFr0xfr-xfr-xb9Gqpi0dc9adbaqaaeGaciGa
% aiaabeqaamaabaabaaGcbaGaaCyqaiabg2da9maadmaabaqbamqabq
% abaaaaaeaacqGHsislcaWGHbWaaSbaaSqaaiaaigdaaeqaaaGcbaGa
% eyOeI0IaamyyamaaBaaaleaacaaIYaaabeaaaOqaaiabl+Uimbqaai
% abgkHiTiaadggadaWgaaWcbaGaamOBaaqabaaakeaacaaIXaaabaGa
% aGimaaqaaiabl+UimbqaaiaaicdaaeaacaaIWaaabaGaeSy8I8eaba
% GaeS47IWeabaGaeSO7I0eabaGaaGimaaqaaiaaicdaaeaacqWIVlct
% aeaacaaIWaaaaaGaay5waiaaw2faaiaacUdacaaMf8UaaCOqaiabg2
% da9maadmaabaqbamqabqqaaaaabaGaaGymaaqaaiaaicdaaeaacqWI
% UlstaeaacaaIWaaaaaGaay5waiaaw2faaaaa!556B!
\[
{\bf{A}} = \left[ {\begin{array}{*{20}c}
   { - a_1 } & { - a_2 } &  \cdots  & { - a_n }  \\
   1 & 0 &  \cdots  & 0  \\
   0 &  \ddots  &  \cdots  &  \vdots   \\
   0 & 0 &  \cdots  & 0  \\
\end{array}} \right];\quad {\bf{B}} = \left[ {\begin{array}{*{20}c}
   1  \\
   0  \\
    \vdots   \\
   0  \\
\end{array}} \right]
\]
with open loop CE:
\[
\det(s\mathbf{I}-\mathbf{A})=s^n+a_1s^{n-1}+\cdots+a_n=0.
\]

\endinput

%%% Local Variables: 
%%% mode: latex
%%% TeX-master: "notes"
%%% End:
\ifslidesonly
\begin{slide}
	\heading{Forced Solution}
   In the control canonical form we have matrices:
% MathType!MTEF!2!1!+-
% faaagaart1ev2aaaKnaaaaWenf2ys9wBH5garuavP1wzZbqedmvETj
% 2BSbqefm0B1jxALjharqqtubsr4rNCHbGeaGqiVu0Je9sqqrpepC0x
% bbL8FesqqrFfpeea0xe9Lq-Jc9vqaqpepm0xbba9pwe9Q8fs0-yqaq
% pepae9pg0FirpepeKkFr0xfr-xfr-xb9Gqpi0dc9adbaqaaeGaciGa
% aiaabeqaamaabaabaaGcbaGaaCyqaiabg2da9maadmaabaqbamqabq
% abaaaaaeaacqGHsislcaWGHbWaaSbaaSqaaiaaigdaaeqaaaGcbaGa
% eyOeI0IaamyyamaaBaaaleaacaaIYaaabeaaaOqaaiabl+Uimbqaai
% abgkHiTiaadggadaWgaaWcbaGaamOBaaqabaaakeaacaaIXaaabaGa
% aGimaaqaaiabl+UimbqaaiaaicdaaeaacaaIWaaabaGaeSy8I8eaba
% GaeS47IWeabaGaeSO7I0eabaGaaGimaaqaaiaaicdaaeaacqWIVlct
% aeaacaaIWaaaaaGaay5waiaaw2faaiaacUdacaaMf8UaaCOqaiabg2
% da9maadmaabaqbamqabqqaaaaabaGaaGymaaqaaiaaicdaaeaacqWI
% UlstaeaacaaIWaaaaaGaay5waiaaw2faaaaa!556B!
\[
{\bf{A}} = \left[ {\begin{array}{*{20}c}
   { - a_1 } & { - a_2 } &  \cdots  & { - a_n }  \\
   1 & 0 &  \cdots  & 0  \\
   0 &  \ddots  &  \cdots  &  \vdots   \\
   0 & 0 &  \cdots  & 0  \\
\end{array}} \right];\quad {\bf{B}} = \left[ {\begin{array}{*{20}c}
   1  \\
   0  \\
    \vdots   \\
   0  \\
\end{array}} \right]
\]
with open loop CE:
\[
\det(s\mathbf{I}-\mathbf{A})=s^n+a_1s^{n-1}+\cdots+a_n=0.
\]

\endinput

%%% Local Variables: 
%%% mode: latex
%%% TeX-master: "notes"
%%% End:
\end{slide}
\fi


 
\subsection*{Example 1}

\textbf{Problem}: Find the solution of:
% MathType!MTEF!2!1!+-
% faaagaart1ev2aaaKnaaaaWenf2ys9wBH5garuavP1wzZbqedmvETj
% 2BSbqefm0B1jxALjharqqtubsr4rNCHbGeaGqiVu0Je9sqqrpepC0x
% bbL8FesqqrFfpeea0xe9Lq-Jc9vqaqpepm0xbba9pwe9Q8fs0-yqaq
% pepae9pg0FirpepeKkFr0xfr-xfr-xb9Gqpi0dc9adbaqaaeGaciGa
% aiaabeqaamaabaabaaGcbaWaaSaaaeaacaWGKbGaaCiEaaqaaiaads
% gacaWG0baaaiabg2da9maadmaabaqbamqabiGaaaqaaiabgkHiTiaa
% iodaaeaacqGHsislcaaIYaaabaGaaGymaaqaaiaaicdaaaaacaGLBb
% GaayzxaaGaaCiEaiabgUcaRmaadmaabaqbamqabiqaaaqaaiaaigda
% aeaacaaIWaaaaaGaay5waiaaw2faaiaadwhaaaa!4013!
\[
\frac{{d{\bf{x}}}}{{dt}} = \left[ {\begin{array}{*{20}c}
   { - 3} & { - 2}  \\
   1 & 0  \\
\end{array}} \right]{\bf{x}} + \left[ {\begin{array}{*{20}c}
   1  \\
   0  \\
\end{array}} \right]u
\]
given $\mathbf{x}_0=[1,\ 1]^T$, and $u = 1$ for $t\ge 0$.

\textbf{SOLUTION}: The state matrix $\mathbf{A}$ is the same as Example 4 from Lecture 17, so the transformation matrix is:
% MathType!MTEF!2!1!+-
% faaagaart1ev2aaaKnaaaaWenf2ys9wBH5garuavP1wzZbqedmvETj
% 2BSbqefm0B1jxALjharqqtubsr4rNCHbGeaGqiVu0Je9sqqrpepC0x
% bbL8FesqqrFfpeea0xe9Lq-Jc9vqaqpepm0xbba9pwe9Q8fs0-yqaq
% pepae9pg0FirpepeKkFr0xfr-xfr-xb9Gqpi0dc9adbaqaaeGaciGa
% aiaabeqaamaabaabaaGcbaGaaCivaiabg2da9iabg2da9maadmaaba
% qbamqabiGaaaqaaiaaigdaaeaacaaIXaaabaGaeyOeI0IaaGymaaqa
% aiabgkHiTiaaicdacaGGUaGaaGynaaaaaiaawUfacaGLDbaaaaa!3935!
\[
{\bf{T}} = \left[ {\begin{array}{*{20}c}
   1 & 1  \\
   { - 1} & { - 0.5}  \\
\end{array}} \right]
\]
with an inverse:
% MathType!MTEF!2!1!+-
% faaagaart1ev2aaaKnaaaaWenf2ys9wBH5garuavP1wzZbqedmvETj
% 2BSbqefm0B1jxALjharqqtubsr4rNCHbGeaGqiVu0Je9sqqrpepC0x
% bbL8FesqqrFfpeea0xe9Lq-Jc9vqaqpepm0xbba9pwe9Q8fs0-yqaq
% pepae9pg0FirpepeKkFr0xfr-xfr-xb9Gqpi0dc9adbaqaaeGaciGa
% aiaabeqaamaabaabaaGcbaGaaCivamaaCaaaleqabaGaeyOeI0IaaG
% ymaaaakiabg2da9iabg2da9maadmaabaqbamqabiGaaaqaaiabgkHi
% TiaaigdaaeaacqGHsislcaaIYaaabaGaaGOmaaqaaiaaikdaaaaaca
% GLBbGaayzxaaaaaa!39A7!
\[
{\bf{T}}^{ - 1}  = \left[ {\begin{array}{*{20}c}
   { - 1} & { - 2}  \\
   2 & 2  \\
\end{array}} \right]
\]
and with corresponding eigenvalues $\lambda_1=-1$ and $\lambda_2=-2$.
 
Therefore the state transition matrix is:
\begin{eqnarray*}
\mathbf{\phi}(t) & = & e^{\mathbf{A}t}=\mathbf{T}e^{\mathbf{\Lambda}t}\mathbf{T}^{-1} \\
	             & = & \left[ {\begin{array}{*{20}c}
	   1 & 1  \\
	   { - 1} & { - 0.5}  \\
	\end{array}} \right]\left[ {\begin{array}{*{20}c}
	   {e^{ - t} } & 0  \\
	   0 & {e^{ - 2t} }  \\
	\end{array}} \right]\left[ {\begin{array}{*{20}c}
	   { - 1} & { - 2}  \\
	   2 & 2  \\
	\end{array}} \right] \\
	             & = & \left[ {\begin{array}{*{20}c}
	   1 & 1  \\
	   { - 1} & { - 0.5}  \\
	\end{array}} \right]\left[ {\begin{array}{*{20}c}
	   {-e^{ - t} } & {-2e^{ - t} }  \\
	   {2e^{ - 2t} } & {2e^{ - 2t} }  \\
	\end{array}} \right] \\
	             & = & \left[ {\begin{array}{*{20}c}
	   \left\{ {-e^{ - t} + 2e^{-2t}} \right\} & \left\{ {-2e^{ - t} + 2e^{-2t}} \right\}  \\
	   \left\{ {e^{ - t} - e^{-2t}} \right\} & \left\{ {2e^{ - t} - e^{-2t}} \right\}  \\
	\end{array}} \right] \\
\end{eqnarray*}

 
The solution is:
\[
\mathbf{x}(t) = \mathrm{term}_1+\int_0^t\mathrm{term}_2 d\tau.
\]
where $\mathrm{term}_1=\mathbf{\phi}(t)\mathbf{x}_0=e^{\mathbf{A}t}\mathbf{x}_0$.
\begin{eqnarray*}
\mathrm{term}_1 & = & \left[ {\begin{array}{*{20}c}
	   \left\{ {-e^{ - t} + 2e^{-2t}} \right\} & \left\{ {-2e^{ - t} + 2e^{-2t}} \right\}  \\
	   \left\{ {e^{ - t} - e^{-2t}} \right\} & \left\{ {2e^{ - t} - e^{-2t}} \right\}  \\
	\end{array}} \right] \left[ {\begin{array}{*{20}c}
	   1  \\
	   1  \\
	\end{array}} \right] \\
	& = & \left[ {\begin{array}{*{20}c}
	   {\left\{ {\left. { - 3e^{ - t}  + 4e^{ - 2t} } \right\}} \right.}  \\
	   {\left\{ {\left. {3e^{ - t}  - 2e^{ - 2t} } \right\}} \right.}  \\
	\end{array}} \right]
\end{eqnarray*}
and
\begin{eqnarray*}
	{\rm{term}}_2  & = & e^{{\bf{A}}(t - \tau )} {\bf{B}}u = e^{{\bf{A}}(t - \tau )} \left[ {\begin{array}{*{20}c}
	   1  \\
	   0  \\
	\end{array}} \right] \times 1 \\
 & = & \left[ {\begin{array}{*{20}c}
   {\left. {\left\{ { - e^{ - (t - \tau )}  + 2e^{ - 2(t - \tau )} } \right.} \right\}}  \\
   {\left. {\left\{ {e^{ - (t - \tau )}  - e^{ - 2(t - \tau )} } \right.} \right\}}  \\
\end{array}} \right]
\end{eqnarray*}

\begin{eqnarray*}
	\int_0^t {{\rm{term}}_2 d\tau }  & = & \left[ {\begin{array}{*{20}c}
	   {\int_0^t {\left. {\left\{ { - e^{ - (t - \tau )}  + 2e^{ - 2(t - \tau )} } \right.} \right\}d\tau } }  \\
	   {\int_0^t {\left. {\left\{ {e^{ - (t - \tau )}  - e^{ - 2(t - \tau )} } \right.} \right\}d\tau } }  \\
	\end{array}} \right] \\
	& = & 
	  \left[ {\begin{array}{*{20}c}
	   {\left. {\left\{ { - e^{ - (t - \tau )}  + e^{ - 2(t - \tau )} } \right.} \right\}}  \\
	   {\left. {\left\{ {e^{ - (t - \tau )}  - 0.5e^{ - 2(t - \tau )} } \right.} \right\}}  \\
	\end{array}} \right]_{\tau  = 0}^t  \\ 
	 & = & \left[ {\begin{array}{*{20}c}
	   { - 1 + e^{ - t}  + 1 - e^{ - 2t} }  \\
	   { - 1 + e^{ - t}  - 0.5 + 0.5e^{ - 2t} }  \\
	\end{array}} \right] \\ 
	 &  = & \left[ {\begin{array}{*{20}c}
	   {e^{ - t}  - e^{ - 2t} }  \\
	   {0.5 - e^{ - t}  + 0.5e^{ - 2t} }  \\
	\end{array}} \right]
\end{eqnarray*}

Combining $\mathrm{term}_1$ and $\int_0^t \mathrm{term}_2 d\tau$ we find the total, forced, response:
% MathType!MTEF!2!1!+-
% faaagaart1ev2aaaKnaaaaWenf2ys9wBH5garuavP1wzZbqedmvETj
% 2BSbqefm0B1jxALjharqqtubsr4rNCHbGeaGqiVu0Je9sqqrpepC0x
% bbL8FesqqrFfpeea0xe9Lq-Jc9vqaqpepm0xbba9pwe9Q8fs0-yqaq
% pepae9pg0FirpepeKkFr0xfr-xfr-xb9Gqpi0dc9adbaqaaeGaciGa
% aiaabeqaamaabaabaaGcbaGaaCiEaiaacIcacaWG0bGaaiykaiabg2
% da9maadmaabaqbamqabiqaaaqaaiabgkHiTiaaiodacaWGLbWaaWba
% aSqabeaacqGHsislcaWG0baaaOGaey4kaSIaaGinaiaadwgadaahaa
% WcbeqaaiabgkHiTiaaikdacaWG0baaaaGcbaGaaG4maiaadwgadaah
% aaWcbeqaaiabgkHiTiaadshaaaGccqGHsislcaaIYaGaamyzamaaCa
% aaleqabaGaeyOeI0IaaGOmaiaadshaaaaaaaGccaGLBbGaayzxaaGa
% ey4kaSYaamWaaeaafaWabeGabaaabaGaamyzamaaCaaaleqabaGaey
% OeI0IaamiDaaaakiabgkHiTiaadwgadaahaaWcbeqaaiabgkHiTiaa
% ikdacaWG0baaaaGcbaGaaGimaiaac6cacaaI1aGaeyOeI0IaaGOmai
% aadwgadaahaaWcbeqaaiabgkHiTiaadshaaaGccqGHRaWkcaaIWaGa
% aiOlaiaaiwdacaWGLbWaaWbaaSqabeaacqGHsislcaaIYaGaamiDaa
% aaaaaakiaawUfacaGLDbaaaaa!5FF5!
\[
{\bf{x}}(t) = \left[ {\begin{array}{*{20}c}
   { - 3e^{ - t}  + 4e^{ - 2t} }  \\
   {3e^{ - t}  - 2e^{ - 2t} }  \\
\end{array}} \right] + \left[ {\begin{array}{*{20}c}
   {e^{ - t}  - e^{ - 2t} }  \\
   {0.5 - 2e^{ - t}  + 0.5e^{ - 2t} }  \\
\end{array}} \right]
\]

Therefore:
% MathType!MTEF!2!1!+-
% faaagaart1ev2aaaKnaaaaWenf2ys9wBH5garuavP1wzZbqedmvETj
% 2BSbqefm0B1jxALjharqqtubsr4rNCHbGeaGqiVu0Je9sqqrpepC0x
% bbL8FesqqrFfpeea0xe9Lq-Jc9vqaqpepm0xbba9pwe9Q8fs0-yqaq
% pepae9pg0FirpepeKkFr0xfr-xfr-xb9Gqpi0dc9adbaqaaeGaciGa
% aiaabeqaamaabaabaaGcbaGaaCiEaiaacIcacaWG0bGaaiykaiabg2
% da9maadmaabaqbamqabiqaaaqaaiabgkHiTiaaikdacaWGLbWaaWba
% aSqabeaacqGHsislcaWG0baaaOGaey4kaSIaaG4maiaadwgadaahaa
% WcbeqaaiabgkHiTiaaikdacaWG0baaaaGcbaGaaGimaiaac6cacaaI
% 1aGaey4kaSIaaGOmaiaadwgadaahaaWcbeqaaiabgkHiTiaadshaaa
% GccqGHsislcaaIXaGaaiOlaiaaiwdacaWGLbWaaWbaaSqabeaacqGH
% sislcaaIYaGaamiDaaaaaaaakiaawUfacaGLDbaaaaa!4C2A!
\[
{\bf{x}}(dt) = \left[ {\begin{array}{*{20}c}
   { - 2e^{ - t}  + 3e^{ - 2t} }  \\
   {0.5 + 2e^{ - t}  - 1.5e^{ - 2t} }  \\
\end{array}} \right]
\]


 

\subsection*{Alternative solution 1}
Using the transformation to normal canonical form:
\[
\frac{d\mathbf{w}}{t}=\mathbf{\Lambda w}+ \mathbf{T}^{-1}\mathbf{B}u
\]
% MathType!MTEF!2!1!+-
% faaagaart1ev2aaaKnaaaaWenf2ys9wBH5garuavP1wzZbqedmvETj
% 2BSbqefm0B1jxALjharqqtubsr4rNCHbGeaGqiVu0Je9sqqrpepC0x
% bbL8FesqqrFfpeea0xe9Lq-Jc9vqaqpepm0xbba9pwe9Q8fs0-yqaq
% pepae9pg0FirpepeKkFr0xfr-xfr-xb9Gqpi0dc9adbaqaaeGaciGa
% aiaabeqaamaabaabaaGcbaWaaSaaaeaacaWGKbGaaC4Daaqaaiaads
% gacaWG0baaaiabg2da9maadmaabaqbamqabiGaaaqaaiabgkHiTiaa
% igdaaeaacaaIWaaabaGaaGimaaqaaiabgkHiTiaaikdaaaaacaGLBb
% GaayzxaaGaaC4DaiabgUcaRmaadmaabaqbamqabiGaaaqaaiabgkHi
% TiaaigdaaeaacqGHsislcaaIYaaabaGaaGOmaaqaaiaaikdaaaaaca
% GLBbGaayzxaaWaamWaaeaafaWabeGabaaabaGaaGymaaqaaiaaicda
% aaaacaGLBbGaayzxaaGaamyDaiaacUdacaaMf8UaamyDaiabg2da9i
% aaigdaaaa!4BE3!
\[
\frac{{d{\bf{w}}}}{{dt}} = \left[ {\begin{array}{*{20}c}
   { - 1} & 0  \\
   0 & { - 2}  \\
\end{array}} \right]{\bf{w}} + \left[ {\begin{array}{*{20}c}
   { - 1} & { - 2}  \\
   2 & 2  \\
\end{array}} \right]\left[ {\begin{array}{*{20}c}
   1  \\
   0  \\
\end{array}} \right]u;\quad u = 1
\]
\[
\frac{dw_1}{dt}=-w_1-1\ \textrm{and}\ \frac{dw_2}{dt}=-2w_2 + 2
\]
% MathType!MTEF!2!1!+-
% faaagaart1ev2aaaKnaaaaWenf2ys9wBH5garuavP1wzZbqedmvETj
% 2BSbqefm0B1jxALjharqqtubsr4rNCHbGeaGqiVu0Je9sqqrpepC0x
% bbL8FesqqrFfpeea0xe9Lq-Jc9vqaqpepm0xbba9pwe9Q8fs0-yqaq
% pepae9pg0FirpepeKkFr0xfr-xfr-xb9Gqpi0dc9adbaqaaeGaciGa
% aiaabeqaamaabaabaaGcbaGaaC4DamaaBaaaleaacaaIWaaabeaaki
% abg2da9iaahsfadaahaaWcbeqaaiabgkHiTiaaigdaaaGccaWH4bWa
% aSbaaSqaaiaaicdaaeqaaOGaeyypa0ZaamWaaeaafaWabeGacaaaba
% GaeyOeI0IaaGymaaqaaiabgkHiTiaaikdaaeaacaaIYaaabaGaaGOm
% aaaaaiaawUfacaGLDbaadaWadaqaauaadeqaceaaaeaacaaIXaaaba
% GaaGymaaaaaiaawUfacaGLDbaacqGH9aqpdaWadaqaauaadeqaceaa
% aeaacqGHsislcaaIZaaabaGaaGinaaaaaiaawUfacaGLDbaaaaa!466E!
\[
{\bf{w}}_0  = {\bf{T}}^{ - 1} {\bf{x}}_0  = \left[ {\begin{array}{*{20}c}
   { - 1} & { - 2}  \\
   2 & 2  \\
\end{array}} \right]\left[ {\begin{array}{*{20}c}
   1  \\
   1  \\
\end{array}} \right] = \left[ {\begin{array}{*{20}c}
   { - 3}  \\
   4  \\
\end{array}} \right]
\]


\begin{eqnarray*}
	w_1 & = & w_{10}e^{-t}-\int_0^t e^{-(t-\tau)} d\tau \\
	    & = & -3e^{-t}-\left[e^{-(t-\tau)}\right]_0^t \\
	    & = & -3e^{-t}-\left[1 - e^{-t}\right] \\
	    & = & -1 -2e^{-t}	
\end{eqnarray*}
 
\begin{eqnarray*}
	w_2 & = & w_{20}e^{-2t}+2\int_0^t e^{-2(t-\tau)} d\tau \\
	    & = & 4e^{-2t}+2\left[0.5e^{-2(t-\tau)}\right]_0^t \\
	    & = & 4e^{-2t}+\left[1 - e^{-2t}\right] \\
	    & = & 1 + 3e^{-2t}	
\end{eqnarray*}

% MathType!MTEF!2!1!+-
% faaagaart1ev2aaaKnaaaaWenf2ys9wBH5garuavP1wzZbqedmvETj
% 2BSbqefm0B1jxALjharqqtubsr4rNCHbGeaGqiVu0Je9sqqrpepC0x
% bbL8FesqqrFfpeea0xe9Lq-Jc9vqaqpepm0xbba9pwe9Q8fs0-yqaq
% pepae9pg0FirpepeKkFr0xfr-xfr-xb9Gqpi0dc9adbaqaaeGaciGa
% aiaabeqaamaabaabaaGcbaGaaCiEaiabg2da9iaahsfacaWH3bGaey
% ypa0ZaamWaaeaafaWabeGacaaabaGaaGymaaqaaiaaigdaaeaacqGH
% sislcaaIXaaabaGaeyOeI0IaaGimaiaac6cacaaI1aaaaaGaay5wai
% aaw2faamaadmaabaqbamqabiqaaaqaaiabgkHiTiaaigdacqGHsisl
% caaIYaGaamyzamaaCaaaleqabaGaeyOeI0IaamiDaaaaaOqaaiaaig
% dacqGHRaWkcaaIZaGaamyzamaaCaaaleqabaGaeyOeI0IaaGOmaiaa
% dshaaaaaaaGccaGLBbGaayzxaaaaaa!49AC!
\[
{\bf{x}} = {\bf{Tw}} = \left[ {\begin{array}{*{20}c}
   1 & 1  \\
   { - 1} & { - 0.5}  \\
\end{array}} \right]\left[ {\begin{array}{*{20}c}
   { - 1 - 2e^{ - t} }  \\
   {1 + 3e^{ - 2t} }  \\
\end{array}} \right]
\]
Therefore
% MathType!MTEF!2!1!+-
% faaagaart1ev2aaaKnaaaaWenf2ys9wBH5garuavP1wzZbqedmvETj
% 2BSbqefm0B1jxALjharqqtubsr4rNCHbGeaGqiVu0Je9sqqrpepC0x
% bbL8FesqqrFfpeea0xe9Lq-Jc9vqaqpepm0xbba9pwe9Q8fs0-yqaq
% pepae9pg0FirpepeKkFr0xfr-xfr-xb9Gqpi0dc9adbaqaaeGaciGa
% aiaabeqaamaabaabaaGcbaGaaCiEaiabg2da9maadmaabaqbamqabi
% qaaaqaaiabgkHiTiaaikdacaWGLbWaaWbaaSqabeaacqGHsislcaWG
% 0baaaOGaey4kaSIaaG4maiaadwgadaahaaWcbeqaaiabgkHiTiaaik
% dacaWG0baaaaGcbaGaaGimaiaac6cacaaI1aGaey4kaSIaaGOmaiaa
% dwgadaahaaWcbeqaaiabgkHiTiaadshaaaGccqGHsislcaaIXaGaai
% OlaiaaiwdacaWGLbWaaWbaaSqabeaacqGHsislcaaIYaGaamiDaaaa
% aaaakiaawUfacaGLDbaaaaa!49D8!
\[
{\bf{x}} = \left[ {\begin{array}{*{20}c}
   { - 2e^{ - t}  + 3e^{ - 2t} }  \\
   {0.5 + 2e^{ - t}  - 1.5e^{ - 2t} }  \\
\end{array}} \right]
\]


 
\subsection*{Alternative solution 2} 
Using the Laplace transform method.

Taking Laplace transforms of the state equations, taking account of the initial conditions:
\[
s\mathbf{X}(s)-\mathbf{x}_0=\mathbf{A}\mathbf{X}(s)+\mathbf{B}U(s)
\]
% MathType!MTEF!2!1!+-
% faaagaart1ev2aaaKnaaaaWenf2ys9wBH5garuavP1wzZbqedmvETj
% 2BSbqefm0B1jxALjharqqtubsr4rNCHbGeaGqiVu0Je9sqqrpepC0x
% bbL8FesqqrFfpeea0xe9Lq-Jc9vqaqpepm0xbba9pwe9Q8fs0-yqaq
% pepae9pg0FirpepeKkFr0xfr-xfr-xb9Gqpi0dc9adbaqaaeGaciGa
% aiaabeqaamaabaabaaGcbaGaaiikaiaadohacaWHjbGaeyOeI0IaaC
% yqaiaacMcacaWHybGaaiikaiaadohacaGGPaGaeyypa0JaaCiEamaa
% BaaaleaacaaIWaaabeaakiabgUcaRiaahkeacaWGvbGaaiikaiaado
% hacaGGPaGaeyypa0ZaamWaaeaafaWabeGabaaabaGaaGymaaqaaiaa
% igdaaaaacaGLBbGaayzxaaGaey4kaSYaamWaaeaafaWabeGabaaaba
% GaaGymaaqaaiaaicdaaaaacaGLBbGaayzxaaWaaSaaaeaacaaIXaaa
% baGaam4Caaaaaaa!4885!
\[
(s{\bf{I}} - {\bf{A}}){\bf{X}}(s) = {\bf{x}}_0  + {\bf{B}}U(s) = \left[ {\begin{array}{*{20}c}
   1  \\
   1  \\
\end{array}} \right] + \left[ {\begin{array}{*{20}c}
   1  \\
   0  \\
\end{array}} \right]\frac{1}{s}
\]
Therefore
% MathType!MTEF!2!1!+-
% faaagaart1ev2aaaKnaaaaWenf2ys9wBH5garuavP1wzZbqedmvETj
% 2BSbqefm0B1jxALjharqqtubsr4rNCHbGeaGqiVu0Je9sqqrpepC0x
% bbL8FesqqrFfpeea0xe9Lq-Jc9vqaqpepm0xbba9pwe9Q8fs0-yqaq
% pepae9pg0FirpepeKkFr0xfr-xfr-xb9Gqpi0dc9adbaqaaeGaciGa
% aiaabeqaamaabaabaaGcbaGaaCiwaiaacIcacaWGZbGaaiykaiabg2
% da9iaacIcacaWGZbGaaCysaiabgkHiTiaahgeacaGGPaWaaWbaaSqa
% beaacqGHsislcaaIXaaaaOWaamWaaeaafaWabeGabaaabaGaaGymai
% abgUcaRmaalaaabaGaaGymaaqaaiaadohaaaaabaGaaGymaaaaaiaa
% wUfacaGLDbaaaaa!3F1F!
\[
{\bf{X}}(s) = (s{\bf{I}} - {\bf{A}})^{ - 1} \left[ {\begin{array}{*{20}c}
   {1 + \frac{1}{s}}  \\
   1  \\
\end{array}} \right]
\]

% MathType!MTEF!2!1!+-
% faaagaart1ev2aaaKnaaaaWenf2ys9wBH5garuavP1wzZbqedmvETj
% 2BSbqefm0B1jxALjharqqtubsr4rNCHbGeaGqiVu0Je9sqqrpepC0x
% bbL8FesqqrFfpeea0xe9Lq-Jc9vqaqpepm0xbba9pwe9Q8fs0-yqaq
% pepae9pg0FirpepeKkFr0xfr-xfr-xb9Gqpi0dc9adbaqaaeGaciGa
% aiaabeqaamaabaabaaGcbaGaaeOtaiaab+gacaqG3bGaaeiiaiaacI
% cacaWGZbGaaCysaiabgkHiTiaahgeacaGGPaWaaWbaaSqabeaacqGH
% sislcaaIXaaaaOGaeyypa0ZaamWaaeaafaWabeGacaaabaGaam4Cai
% abgUcaRiaaiodaaeaacaaIYaaabaGaeyOeI0IaaGymaaqaaiaadoha
% aaaacaGLBbGaayzxaaWaaWbaaSqabeaacqGHsislcaaIXaaaaOGaey
% ypa0ZaaSaaaeaacaaIXaaabaGaaiikaiaadohacqGHRaWkcaaIZaGa
% aiykaiaadohacqGHRaWkcaaIYaaaamaadmaabaqbamqabiGaaaqaai
% aadohaaeaacqGHsislcaaIYaaabaGaaGymaaqaaiaadohacqGHRaWk
% caaIZaaaaaGaay5waiaaw2faaaaa!5355!
\[
{\rm{Now }}(s{\bf{I}} - {\bf{A}})^{ - 1}  = \left[ {\begin{array}{*{20}c}
   {s + 3} & 2  \\
   { - 1} & s  \\
\end{array}} \right]^{ - 1}  = \frac{1}{{(s + 3)s + 2}}\left[ {\begin{array}{*{20}c}
   s & { - 2}  \\
   1 & {s + 3}  \\
\end{array}} \right]
\]
Therefore
\begin{eqnarray*}
	{\bf{X}}(s) & = & \frac{1}{{s^2  + 3s + 2}}\left[ {\begin{array}{*{20}c}
	   s & { - 2}  \\
	   1 & {s + 3}  \\
	\end{array}} \right]\left[ {\begin{array}{*{20}c}
	   {1 + \frac{1}{s}}  \\
	   1  \\
	\end{array}} \right] \\
	            & = & \frac{1}{{(s + 1)(s + 2)}}\left[ {\begin{array}{*{20}c}
	   {s - 1}  \\
	   {\frac{{s^2  + 4s + 1}}{s}}  \\
	\end{array}} \right] \\
	&  = & \left[ {\begin{array}{*{20}c}
	   {\frac{{s - 1}}{{(s + 1)(s + 2)}}}  \\
	   {\frac{{s^2  + 4s + 1}}{{s(s + 1)(s + 2)}}}  \\
	\end{array}} \right]
\end{eqnarray*}

Taking partial fractions:
\[
{\bf{X}}(s) = \left[ {\begin{array}{*{20}c}
   {\frac{{ - 2}}{{s + 1}} + \frac{3}{{s + 2}}}  \\
   {\frac{{0.5}}{s} + \frac{2}{{s + 1}} + \frac{{ - 1.5}}{{s + 2}}}  \\
\end{array}} \right]
\]

Finally, taking inverse Laplace transforms:
\[
{\bf{x}}(t) = \left[ {\begin{array}{*{20}c}
   { - 2e^{ - t}  + 3e^{ - 2t} }  \\
   {0.5 + 2e^{ - t}  - 1.5e^{ - 2t} }  \\
\end{array}} \right]
\]





%----------------------------------------------------------------
% The end of notes
% ----------------------------------------------------------------
\endinput

%%% Local Variables: 
%%% mode: latex
%%% TeX-master: t
%%% End: 
